\documentclass[11pt,a4paper,oneside]{amsart}

\usepackage{enumitem}
\usepackage{dsfont}
\usepackage{amssymb,amsthm,amsmath}
\usepackage{amsfonts}
\usepackage{comment}
\usepackage{mathtools}
\usepackage{mathrsfs}
\usepackage{microtype}
\usepackage{graphicx}
\usepackage[T1]{fontenc}
\usepackage[ngerman]{babel}

\input{xy}
\xyoption{all}


\theoremstyle{definition}
%\newtheorem{blatt}{Blatt}[section]
\newtheorem{aufg}{Aufgabe}[section]
\newtheorem*{lsg}{L{\"o}sung}


\setlength{\parindent}{1.5em}
\setlength{\parskip}{.5ex}

\usepackage{color}

\usepackage[colorlinks,breaklinks]{hyperref}

\DeclareMathOperator{\Part}{Part}
\DeclareMathOperator{\fin}{fin}

%%%% Legendre-Symbol
\makeatletter
\newcommand{\LegendreGap}[2]{%
  \mathchoice
  {{\sbox0{$\genfrac{}{}{0pt}{0}{#1}{#2}$}%
      \vphantom{\copy0}\ooalign{\hidewidth%
        $\vcenter{\moveright\nulldelimiterspace %
          \hbox to\wd0{\hbox{\vrule height 0.4pt width 0.5\wd0\kern 2pt\vrule height 0.4pt width 0.5\wd0}}%
        }$
        \hidewidth\cr $\genfrac{(}{)}{0pt}{0}{#1}{#2}$\cr}}}
  {{\sbox0{$\genfrac{}{}{0pt}{1}{#1}{#2}$}%
      \vphantom{\copy0}\ooalign{\hidewidth%
        $\vcenter{\moveright\nulldelimiterspace %
          \hbox to\wd0{\hbox{\vrule height 0.4pt width 0.5\wd0\kern 2pt\vrule height 0.4pt width 0.5\wd0}}%
        }$
        \hidewidth\cr $\genfrac{(}{)}{0pt}{1}{#1}{#2}$\cr}}}
  {{\sbox0{$\genfrac{}{}{0pt}{2}{#1}{#2}$}%
      \vphantom{\copy0}\ooalign{\hidewidth%
        $\vcenter{\moveright\nulldelimiterspace %
          \hbox to\wd0{\hbox{\vrule height 0.4pt width 0.5\wd0\kern 2pt\vrule height 0.4pt width 0.5\wd0}}%
        }$
        \hidewidth\cr $\genfrac{(}{)}{0pt}{2}{#1}{#2}$\cr}}}
  {{\sbox0{$\genfrac{}{}{0pt}{3}{#1}{#2}$}%
      \vphantom{\copy0}\ooalign{\hidewidth%
        $\vcenter{\moveright\nulldelimiterspace %
          \hbox to\wd0{\hbox{\vrule height 0.4pt width 0.5\wd0\kern 2pt\vrule height 0.4pt width 0.5\wd0}}%
        }$
        \hidewidth\cr$\genfrac{(}{)}{0pt}{3}{#1}{#2}$\cr}}}
}

\newcommand{\Legendre}[2]{
  \mathchoice{\genfrac{(}{)}{0.4pt}{0}{#1}{#2}}
  {\genfrac{(}{)}{0.4pt}{1}{#1}{#2}}
  {\genfrac{(}{)}{0.4pt}{2}{#1}{#2}}
  {\genfrac{(}{)}{0.4pt}{3}{#1}{#2}}
}

\makeatother
%%%%




\newcommand{\field}{\mathbb{F}}
\newcommand{\mmod}{\ \mathrm{mod}\ }
\newcommand{\boldone}{\mathbf{1}}

\DeclareMathOperator{\PW}{PW}
\DeclareMathOperator{\PWM}{PWM}
\DeclareMathOperator{\ggT}{ggT}
\DeclareMathOperator{\kgV}{kgV}
\newcommand{\Primes}{\mathbb{P}}
\newcommand{\setmid}{\;:\;}
\DeclareMathOperator{\Primlength}{PL}

\DeclareMathOperator{\base}{base}
\newcommand{\homsp}{\mathcal X}
\DeclareMathOperator{\Li}{Li}
\DeclareMathOperator{\ord}{ord}
\DeclareMathOperator{\WF}{WF}
\newcommand{\bT}{\mathbf T}
\DeclareMathOperator{\Gen}{Gen}
\newcommand{\redu}{\text{red}}
% orbifolds

\DeclareMathOperator{\Eff}{Eff}
\DeclareMathOperator{\germ}{germ}
\DeclareMathOperator{\Germ}{Germ}
\DeclareMathOperator{\dom}{dom}
\DeclareMathOperator{\ran}{ran}
\DeclareMathOperator{\cod}{cod}
\DeclareMathOperator{\Diff}{Diff}
\DeclareMathOperator{\Emb}{Emb}
\DeclareMathOperator{\Orbmap}{Orb}
\newcommand{\pullback}[2]{\,{}_{#1}\!\!\times_{#2}}
\DeclareMathOperator{\Agr}{Agr}

% algebraische Strukturen

\DeclareMathOperator{\Isom}{Isom}
\DeclareMathOperator{\Hom}{Hom}
\DeclareMathOperator{\Aut}{Aut}
\DeclareMathOperator{\Morph}{Morph}
\DeclareMathOperator{\End}{End}
\DeclareMathOperator{\Bil}{Bil}
\DeclareMathOperator{\QF}{QF}
\DeclareMathOperator{\Op}{Op}
\DeclareMathOperator{\Object}{Ob}
\DeclareMathOperator{\Spann}{Spann}
\DeclareMathOperator{\spann}{span}
\DeclareMathOperator{\ER}{ER}
\DeclareMathOperator{\Rang}{Rang}
\DeclareMathOperator{\Gal}{Gal}
\newcommand{\vs}{\text{vs}}
\newcommand{\Res}{\text{res}}
\DeclareMathOperator{\Var}{Var}
\DeclareMathOperator{\Lie}{Lie}

%Standardmatrizen 

\DeclareMathOperator{\Mat}{Mat}
\DeclareMathOperator{\GL}{GL}
\DeclareMathOperator{\AGL}{AGL}
\DeclareMathOperator{\Gl}{GL}
\DeclareMathOperator{\SL}{SL}
\DeclareMathOperator{\PSL}{PSL}
\DeclareMathOperator{\PGL}{PGL}
\DeclareMathOperator{\PU}{PU}
\DeclareMathOperator{\Sym}{Sym}
\DeclareMathOperator{\Sp}{Sp}
\DeclareMathOperator{\PSp}{PSp}
\DeclareMathOperator{\SO}{SO}
\DeclareMathOperator{\Orth}{O}
\DeclareMathOperator{\Spin}{Spin}
\DeclareMathOperator{\PGamma}{P\Gamma}

\DeclareMathOperator{\diag}{diag}

% Operatoren

\DeclareMathOperator{\Tr}{Tr}
\DeclareMathOperator{\tr}{tr}
\DeclareMathOperator{\rank}{rank}
\DeclareMathOperator{\grad}{grad}
\DeclareMathOperator{\Div}{div}
\DeclareMathOperator{\Ima}{Im}
\DeclareMathOperator{\Rea}{Re}
\DeclareMathOperator{\sgn}{sgn}

\DeclareMathOperator{\supp}{supp}
\DeclareMathOperator{\Span}{span}
\DeclareMathOperator{\Bild}{Bild}

\DeclareMathOperator{\pr}{pr}
\DeclareMathOperator{\res}{res}
% besondere Matrizen

\DeclareMathOperator{\red}{Red}
\DeclareMathOperator{\I}{I}
%\newcommand{\I}{\mathds{1}}


% Lietheorie

\DeclareMathOperator{\Exp}{Exp}
\DeclareMathOperator{\Ad}{Ad}
\DeclareMathOperator{\ad}{ad}
\DeclareMathOperator{\AD}{\textbf{Ad}}
\DeclareMathOperator{\Ind}{Ind}
\DeclareMathOperator{\fsl}{\mathfrak{sl}}

% Wirkungen

\DeclareMathOperator{\Stab}{Stab}
\DeclareMathOperator{\Fix}{Fix}
\DeclareMathOperator{\VS}{VS}

% Masse
\DeclareMathOperator{\HB}{HB}
\DeclareMathOperator{\cusp}{cusp}

% symbolische Dynamik

\DeclareMathOperator{\abs}{abs}
\DeclareMathOperator{\Ext}{ext}
\DeclareMathOperator{\Int}{int}
\DeclareMathOperator{\height}{ht}
\DeclareMathOperator{\cl}{cl}
\DeclareMathOperator{\Cod}{Cod}
\DeclareMathOperator{\NC}{NC}
\DeclareMathOperator{\D}{D}
\DeclareMathOperator{\MD}{MD}
\DeclareMathOperator{\Prim}{P}
\DeclareMathOperator{\Rel}{Rel}
\DeclareMathOperator{\HT}{HT}
\DeclareMathOperator{\vb}{vb}
\DeclareMathOperator{\vc}{vc}
\DeclareMathOperator{\bd}{bd}
\DeclareMathOperator{\BS}{BS}
\DeclareMathOperator{\CS}{CS}
\DeclareMathOperator{\DS}{DS}
\DeclareMathOperator{\IS}{IS}
\DeclareMathOperator{\NIC}{NIC}
\DeclareMathOperator{\Sides}{Sides}
\DeclareMathOperator{\Seq}{Seq}
\DeclareMathOperator{\cyl}{cyl}
\DeclareMathOperator{\IN}{IN}
\DeclareMathOperator{\sym}{sym}
\DeclareMathOperator{\Per}{Per}
\DeclareMathOperator{\ES}{ES}

\newcommand{\hg}{\overline \h^g}
\newcommand{\chg}{\cl_{\overline \h^g}}
\newcommand{\bhg}{\partial_g}

\newcommand{\dg}{\overline D^g}
\newcommand{\fcs}{\CS'\hspace{-.9mm}\big(\wt\fch_{\choices,\shmap}\big)}

\newcommand{\rueck}{\hspace{-.9mm}}

\newcommand{\fd}{\mc F}
\newcommand{\pch}{\mc A}
\newcommand{\fpch}{\mathbb A}
\newcommand{\ch}{\mc B}
\newcommand{\fch}{\mathbb B}
\newcommand{\choices}{\mathbb S}
\newcommand{\shmap}{\mathbb T}
\newcommand{\leer}{\Diamond}

\newcommand{\rd}{\text{red}}
\newcommand{\st}{\text{st}}
\newcommand{\all}{\text{all}}
\newcommand{\bk}{\text{bk}}
\newcommand{\tw}{\text{tw}}
\newcommand{\dec}{\text{dec}}
\newcommand{\parab}{\text{par}}
\newcommand{\dyn}{\text{dyn}}
% Buchstaben

\newcommand\F{\mathbb{F}}
\newcommand\N{\mathbb{N}}
\newcommand\Q{\mathbb{Q}}
\newcommand\R{\mathbb{R}}
\newcommand\Z{\mathbb{Z}}
\newcommand\C{\mathbb{C}}
\newcommand\dD{\mathbb{D}}
\newcommand{\h}{\mathbb{H}}
\newcommand{\mP}{\mathbb{P}}
\newcommand\T{\mathbb{T}}

\newcommand{\mc}[1]{\mathcal #1}
\newcommand{\mf}[1]{\mathfrak #1}
\newcommand{\mb}[1]{\mathbb #1}
\newcommand{\mft}[2]{\mathfrak #1\mathfrak #2}
\newcommand{\wt}{\widetilde}
\newcommand{\wh}{\widehat}

\newcommand{\eps}{\varepsilon}
\newcommand\gG{\Gamma}
\newcommand\gd{\delta}


% speziell fuer SL2R
\DeclareMathOperator{\Pe}{P}
\DeclareMathOperator{\spec}{spec}
\DeclareMathOperator{\dist}{dist}
\DeclareMathOperator{\lsp}{lsp}
\DeclareMathOperator{\grp}{grp}
\DeclareMathOperator{\dvol}{dvol}
\DeclareMathOperator{\vol}{vol}
\DeclareMathOperator{\im}{im}
\DeclareMathOperator{\Orb}{\mc O}
\DeclareMathOperator{\mult}{mult}
\DeclareMathOperator{\Arcosh}{Arcosh}
\DeclareMathOperator{\arccot}{arccot}



% Koecher

\DeclareMathOperator{\RMod}{R-mod}
\DeclareMathOperator{\Ob}{Ob}

% FT
\DeclareMathOperator{\ind}{ind}

%BLZ
\DeclareMathOperator{\FE}{FE}

% Sonstiges

\DeclareMathOperator{\id}{id}
\DeclareMathOperator{\M}{M}
\DeclareMathOperator{\Graph}{graph}
\DeclareMathOperator{\Fct}{Fct}
\DeclareMathOperator{\MCF}{MCF}
\DeclareMathOperator{\vN}{vN}

\DeclareMathOperator{\esssup}{ess\,sup}

\newcommand{\sceq}{\mathrel{\mathop:}=}
\newcommand{\seqc}{\mathrel{=\mkern-4.5mu{\mathop:}}}

\newcommand{\mat}[4]{\begin{pmatrix} #1&#2\\#3&#4\end{pmatrix}}
\newcommand{\bmat}[4]{\begin{bmatrix} #1&#2\\#3&#4\end{bmatrix}}
\newcommand{\textmat}[4]{\left(\begin{smallmatrix} #1&#2 \\ #3&#4
\end{smallmatrix}\right)}
\newcommand{\textbmat}[4]{\left[\begin{smallmatrix} #1&#2 \\ #3&#4
\end{smallmatrix}\right]}
\newcommand{\vvek}[2]{\begin{pmatrix} #1 \\ #2 \end{pmatrix}}
\newcommand{\hvek}[2]{\begin{pmatrix} #1 & #2 \end{pmatrix}}

\newcommand\ie{\mbox{i.\,e., }}
\newcommand\eg{\mbox{e.\,g., }}
\newcommand\wrt{\mbox{w.\,r.\,t.\@ }}
\newcommand\Wlog{\mbox{w.\,l.\,o.\,g.\@ }}
\newcommand\resp{\mbox{resp.\@}}

\newcommand\abstand{\vspace{0.5cm}}
\newcommand\negab{\vspace{-0.5cm}}

\DeclareMathOperator{\ggt}{ggt}

\newcommand\mminus{\!\smallsetminus\!}

%%%%%%%%%%%%%%%%%%%%%%%%%%%%%%%%%%%%%%%%%%%%%%%%%%%%%%%%%%%%%%

%\usepackage[textwidth=13cm]{geometry}
%\usepackage[notref,notcite]{showkeys}




\begin{document}

\title[Analysis~1, L\"osungen]{Analysis~1 \\ L\"osungen der \"Ubungsaufgaben\\ Winter~2021/22}
\author{H\"orer*innen der Vorlesung}
\address{H\"orer*innen der Vorlesung \glqq Analysis~1\grqq{} von Prof.~Anke Pohl, Universit\"at Bremen}
\date{Winter~2021/22}


\begin{abstract}
Dieses ist die Sammlung von L\"osungen der \"Ubungsaufgaben zur Vorlesung \glqq Analysis~1\grqq{} im Wintersemester~2021/22 an der Universit\"at Bremen, die von den H\"orer*innen der Vorlesung gemeinsam erstellt wird. Die Vorlesung wird von Prof.~Dr.~Anke Pohl gehalten.
\end{abstract}


\maketitle

\tableofcontents

\newpage 
\section*{Schema}

So geht das Schema:

\begin{center}
\begin{minipage}{.6\textwidth}
\begin{verbatim}

Das ist eine Umgebung für Aufgaben:

\begin{aufg}
 Hier steht die Aufgabe.
\end{aufg}

Das ist eine Umgebung für Lösungen:

\begin{lsg}
 Hier steht die Lösung.
\end{lsg}
\end{verbatim}
\end{minipage}
\end{center}

\bigskip 

Wichtig:
\begin{itemize}
 \item Keine Umlaute tippen, sondern immer durch LaTeX-Code erzeugen, z.B.\@ \verb!\"a! f\"ur \"a.
 \item Keine packages verwenden, die nicht in texlive-full enthalten sind. 
 \item Keine Leerzeichen, Umlaute, Sonderzeichen in Dateinamen!
\end{itemize}


\newpage
\section*{Blatt 1}

\setcounter{blatt}{1}

\begin{aufg}[2 Punkte]\mbox{ }
\begin{enumerate}[label=$\mathrm{(\roman*)}$, ref=$\mathrm{\roman*}$]
\item Lesen Sie im Stud.IP-Forum den Teil zu git und gitlab durch, schauen Sie die verlinkten Videos, lesen Sie (mindestens einen Teil) der Anleitungen.
\item Besorgen Sie sich eine Freischaltung für den gitlab-Server im FB~3. Im Stud.IP-Forum finden Sie eine Anleitung.
\item Lassen Sie sich zum git repository f\"ur die Analysis~1 eintragen. (Die ersten werden von uns eingetragen. Sprechen Sie z.B.\@ Ihre \"Ubungsleiter*innen an. Alle Eingetragenen k\"onnen dann weitere Personen eintragen, sobald diese Teil~(i) erledigt haben. Das geht bei ``Project Information'', dann ``Members''. Dabei Status ``maintainer'' ausw\"ahlen.) 
\end{enumerate}
\end{aufg}

\bigskip

\begin{aufg}[6 Punkte]
Es seien $A$, $B$ und $C$ Mengen. Beweisen Sie: 
\begin{enumerate}[label=$\mathrm{(\roman*)}$, ref=$\mathrm{\roman*}$]
 \item Es gilt 
 \[
  A\cup B = A\cap B \quad\Leftrightarrow\quad  A=B\,.
 \]
 \item Es gilt 
 \[
  A \cap (B \cup C) = (A\cap B) \cup (A\cap C)\,.
 \]
 \item Es gilt 
 \[
  A \cup (B \cap C) = (A\cup B) \cap (A\cup C)\,.
 \]
\end{enumerate}
\end{aufg}
 
\bigskip 

\begin{lsg}
 
\end{lsg} 

\bigskip


\begin{aufg}[6 Punkte]
Sind die folgenden Relationen Funktionen?
\begin{enumerate}[label=$\mathrm{(\roman*)}$, ref=$\mathrm{\roman*}$]
\item $f \coloneqq \{(n,m)\in \N\times\N \mid n=m^2\}$
\item $f \coloneqq \{(x,y)\in \R\times\R \mid x=y^2\}$
\item $f \coloneqq \{(n,m)\in \N\times\N \mid n^2=1+m\}$
\end{enumerate}
F\"ur diese Aufgabe d\"urfen Sie alle bekannten Eigenschaften f\"ur $\N=\{1,2,3,\ldots\}$ und $\R$ verwenden.
\end{aufg}

\bigskip 

\begin{lsg}
 
\end{lsg}

\bigskip

\begin{aufg}[6 Punkte]
Bei einer Bev\"olkerungsbefragung wird eine Frau in ihrem Haus befragt, wer dort wohne. Sie antwortet: \glqq Mein Mann und ich mit unseren drei T\"ochtern.\grqq{} Auf die Frage nach dem Alter der T\"ochter antwortet sie: \glqq Multipliziert man ihr Alter erh\"alt man~$36$. Die Summe ihrer Alter ist unsere Hausnummer.\grqq{} Der Befrager liest die Hausnummer ab, denkt kurz nach und sagt dann: \glqq Mit den Informationen kann man die Alter ihrer T\"ochter nicht bestimmen.\grqq{}. Die Frau antwortet: \glqq Ja, da haben Sie Recht. Dann sage ich Ihnen noch, dass meine \"alteste Tochter gerade in ihrem Zimmer schl\"aft.\grqq{} Der Befrager antwortet: \glqq Dankesch\"on!\grqq{} und geht gl\"ucklich seines Weges. 

Wie alt sind die T\"ochter? (Selbstverst\"andlich mit Begr\"undung.)

\end{aufg}
 
\bigskip 

\begin{lsg}
 
\end{lsg}

\newpage
\section{Blatt}


\begin{aufg}[6 Punkte]
Beweise die folgenden Identit\"aten f\"ur $a,b,c,d \in Z$:
\begin{enumerate}[label=$\mathrm{(\roman*)}$, ref=$\mathrm{\roman*}$]
\item $a+(b-c)=(a+b)-c$
\item $-(b-a)=a-b$
\item $a-(b-c)=(a+c)-b$
\item $-(a+b)=-a-b$
\end{enumerate}
\end{aufg}

\bigskip

\begin{lsg}[Leander Sims, Jütte Apel]
Wir kürzen Kommutativität mit Ko. und Assoziativität mit As. ab.
\begin{enumerate}[label=$\mathrm{(\roman*)}$, ref=$\mathrm{\roman*}$]
\item
    Es ist $ a+(b-c) $ die eindeutige L\"osung von 
    \[c+x =(a+b).\]
    Wir zeigen, dass auch $(a+b)-c$ die Gleichung $c+x =a+b$ l\"ost.\\
    Beweis:
    \begin{align*}
	    && \quad  c+((a+b)-c) \overset{\text{Ko.}}{=} & \quad ((a+b)-c)+c\\
    	&&\overset{\text{2.8ii}}{=} & \quad ((a+b)+(-c))+c\\
	    &&\overset{\text{As.}}{=} & \quad (a+b)+((-c)+c)\\
	    &&\overset{\text{2.8i}}{=} & \quad (a+b)+0\\
    	&&\overset{\text{2.4}}{=} & \quad (a+b)
\end{align*}
Wegen Eindeutigkeit gilt $(a+b)-c=a+(b-c)$.
\item
    Es ist $ -(b-a) $ die eindeutige L\"osung von 
    \[b+x =a.\]
	Wir zeigen, dass auch $(a-b)$ die Gleichung $b+x =a$ l\"ost.\\
	Beweis:
	\begin{align*}
		&& \quad  b+(a-b)\overset{\text{Ko.}}{=}& \quad  (a-b)+b\\
		&&\overset{\text{2.8ii}}{=} & \quad (a+(-b))+b\\
		&&\overset{\text{As.}}{=} & \quad a+((-b)+b)\\
		&&\overset{\text{2.8i}}{=} & \quad a +0\\
		&&\overset{\text{2.4}}{=} & \quad a
	\end{align*}
	Wegen Eindeutigkeit gilt $-(b-a)=a-b$.
\item
    Es ist $a-(b-c)$ die eindeutige L\"osung von 
	\[ b+x=(a+c). \]
	Wir zeigen, dass auch $(a+c)-b$ die Gleichung $b+x=a+c$ l\"ost.\\
	Beweis:
	\begin{align*}
		&& \quad b+((a+c)-b)\overset{\text{Ko.}}{=} & \quad ((a+c)-b)+b\\
		&&\overset{\text{2.8ii}}{=} & \quad ((a+c)+(-b))+b\\
		&&\overset{\text{As.}}{=} & \quad (a+c)+((-b)+b)\\
		&&\overset{\text{2.8i}}{=} & \quad (a+c)+0\\
		&&\overset{\text{2.4}}{=} & \quad (a+c)
	\end{align*}\\
	Wegen Eindeutigkeit gilt $a-(b-c)=(a+c)-b$.
\item
    Es ist $-(a+b)$ die eindeutige L\"osung von
	\[b+x=(-a).\]
	Wir zeigen, dass auch $ (-a-b) $ die Gleichung $b+x=(-a)$ l\"ost.\\
	Beweis:
	\begin{align*}
		&&\quad b+(-a-b)\overset{\text{2.8ii}}{=} & \quad b+((-a)+(-b))\\
		&&\overset{\text{Ko.}}{=} & \quad -b+((-b)+(-a))\\
		&&\overset{\text{As.}}{=} & \quad (-b+(-b))+(-a)\\
		&&\overset{\text{2.8i}}{=} & \quad 0+(-a)\\
		&&\overset{\text{2.4}}{=} & \quad (-a)
		\end{align*}
	Wegen Eindeutigkeit gilt $-(a+b)=-a-b$.
\end{enumerate}
\end{lsg}

\bigskip

\begin{aufg}[6 Punkte]
Zeigen Sie: $\left( \{a,b\}, + , \cdot \right)$ mit $a\not=b$ und 
\begin{center}
\begin{tabular}{c|cc}
 $+$ & $a$ & $b$
 \\ \hline
 $a$ & $a$ & $b$
 \\
 $b$ & $b$ & $a$
\end{tabular}
\qquad
\begin{tabular}{c|cc}
 $\cdot$ & $a$ & $b$
 \\ \hline
 $a$ & $a$ & $a$
 \\
 $b$ & $a$ & $b$
\end{tabular}
\end{center}
ist ein K\"orper. Gibt es eine Anordnung (mit Beweis!)?
\end{aufg}
 
\bigskip

\begin{lsg}
\end{lsg}

\bigskip


\begin{aufg}[6 Punkte]
Beweisen Sie die folgenden Identit\"aten f\"ur $a,b,c,d \in Z$, $b\neq 0$, $d\neq 0$ 
\begin{enumerate}[label=$\mathrm{(\roman*)}$, ref=$\mathrm{\roman*}$]
\item $\frac{a}{b} \cdot \frac{c}{d} = \frac{ac}{bd}$, 
\item $\frac{\frac{a}{b}}{\frac{c}{d}} = \frac{ad}{bc}$ f\"ur $c\not=0$,
\item $c \frac{a}{b} = \frac{ca}{b}$.
\end{enumerate}
\end{aufg}

\bigskip

\begin{comment}
\newcommand\Asseq{\stackrel{\mathclap{\normalfont\fontsize{4}\mbox{Ass.}}}{=}}
\newcommand\KommAsseq{\stackrel{\mathclap{\normalfont\fontsize{4}\mbox{Komm. & Ass.}}}{=}}
\newcommand\Defeq{\stackrel{\mathclap{\normalfont\fontsize{4}\mbox{Def. Quotient}}}{=}}
\newcommand\ieq{\stackrel{\mathclap{\normalfont\fontsize{4}\mbox{(i)}}}{=}}
\end{comment}


\newcommand\Asseq{\stackrel{\text{Ass.}}{=}}
\newcommand\KommAsseq{\stackrel{\text{Komm. \& Ass.}}{=}}
\newcommand\Defeq{\stackrel{\text{Def. Quotient}}{=}}
\newcommand\ieq{\stackrel{\text{(i)}}{=}}

\begin{lsg}[Neila Fettous und Manuel Dammert]
\mbox{ }
\begin{enumerate}[label=$\mathrm{(\roman*)}$, ref=$\mathrm{\roman*}$]
\item z.Z: $\frac{a}{b} \cdot \frac{c}{d} = \frac{ac}{bd}$\\
$(bd) \cdot x = ac$, x = $\frac{ac}{bd}$ l\"ost nach Def. des Quotienten die Gleichung.
$\frac{a}{b} \cdot \frac{c}{d}$ l\"ost auch, denn 
\[
(bd) \cdot (\frac{a}{b} \cdot \frac{c}{d}) \Asseq 
bd \cdot \frac{a}{b} \cdot \KommAsseq (b \frac{a}{b}) \cdot (d\frac{c}{d}) \Defeq ac \qed
\]
\item z.Z: $\frac{\frac{a}{b}}{\frac{c}{d}} = \frac{ad}{bc}$ f\"ur $c\not=0$\\
$(\frac{c}{d} \cdot x = \frac{a}{b}$, $x = \frac{\frac{a}{b}}{\frac{c}{d}}$ l\"ost nach Def. des Quotienten die Gleichung. $\frac{ac}{bd}$ l\"ost auch, denn
\[
(\frac{c}{d}) \cdot (\frac{ad}{bc}) \Asseq 
\frac{c}{d} \cdot \frac{ad}{bd} \ieq \frac{a}{b} \qed
\]
\item z.Z: $c\frac{a}{b} = \frac{ca}{b}$\\
$b \cdot x = ca$, $x = \frac{ca}{b}$ l\"ost per Def. des Quotienten die Gleichung, $c\frac{a}{b}$ l\"ost auch, denn
\[
b \cdot (c\frac{a}{b}) \Asseq 
b \cdot c \cdot \frac{a}{b} \KommAsseq c \cdot (b \cdot \frac{a}{b} \Defeq ca \qed
\]
\end{enumerate}
\end{lsg}

\bigskip


\begin{aufg}[4 Punkte]
\begin{enumerate}[label=$\mathrm{(\roman*)}$, ref=$\mathrm{\roman*}$]
\item Was ist anschaulich der Unterschied zwischen~$\R$ und~$\Q$? (Nutzen Sie Ihr Schulwissen zu~$\R$ und~$\Q$.)
\item Klassische Mousse au chocolat besteht aus 3-4 Zutaten. Ist die Reihenfolge des Zusammenf\"ugens der Zutaten egal oder nicht? In anderen Worten, erf\"ullt die Zubereitung das Assoziativit\"atsaxiom?
\end{enumerate}
\end{aufg}
 
\bigskip

\begin{lsg}[Pia Blanke, Pia Hovemann]\mbox{ }
\begin{enumerate}[label=$\mathrm{(\roman*)}$, ref=$\mathrm{\roman*}$]
\item ges.: Der Unterschied zwischen $\R$ und $\Q$ 
$\Q$  ist die Menge aller rationalen Zahlen und enth\"alt alle positiven und negativen Br\"uche, abbrechende Dezimalbr\"uche und periodische Dezimalbr\"uche.
$\R$ ist die Menge der reellen Zahlen und beinhaltet die rationalen Zahlen, sowie die irrationalen Zahlen.
Der Unterschied zwischen den beiden Mengen liegt also darin, dass $\R$ zus\"atzlich zu allen Elementen aus $\Q$  auch irrationalen Zahlen wir Wurzeln enth\"alt.
%
\item zz.: Erf\"ullt die Zubereitung der klassischen Mousse au Chocolat das Assoziativit\"atsaxiom?
Klassische Mousse au Chocolat besteht in der Regel aus geschmolzener dunkler Schokolade, Eiern (wobei Eiwei{\ss} und Eigelb getrennt voneinander verarbeitet werden) und Puderzucker. Die Mengen sind zur Beantwortung der Frage unerheblich und werden deswegen hier nicht explizit benannt.
Bei der Zubereitung werden zun\"achst die Schokolade temperiert, dann die Eigelbe aufgeschlagen und die Eiwei{\ss} zu Eischnee verarbeitet. Der Puderzucker wird zum Eischnee hinzugef\"ugt, wenn dieser die richtige Konsistenz erreicht hat, da der Eischnee auf diese Weise fixiert werden kann. Hier wird also schon erkenntlich, dass es einen Unterschied macht, wenn der Puderzucker zu einem anderen Zeitpunkt eingesetzt/hinzugef\"ugt wird. 
Zus\"atzlich ist es wichtig, dass die Eigelbe unter den Eischnee gehoben werden, bevor die Schokoladenmasse darunter gemischt wird, um die Leichtigkeit zu erhalten. Die Eigelbe direkt unter die Schokolade zu r\"uhren, w\"urde zu einer erheblichen Verfestigung der Masse f\"uhren, was nicht dem luftigen Dessert entsprechen w\"urde, dass Mousse au Chocolat sein soll.
Es gilt also mit den Variablen Schokolade $\coloneqq S$, Eigelb $\coloneqq E_{g}$ , und mit Puderzucker vermischter Eischnee $\coloneqq E_{p}$
$(E_{p} + E_{g}) + S \neq E_{p} + (E_{g} + S)$
Das Assoziativit\"atsaxiom gilt bei der Zubereitung von Mousse au Chocolat nicht.
\end{enumerate}
\end{lsg}

\newpage
\section{Blatt}

\begin{aufg}[6 Punkte]\label{kleiner}
Seien $a,b\in Z$. Beweisen Sie: 
\begin{enumerate}[label=$\mathrm{(\roman*)}$, ref=$\mathrm{\roman*}$]
\item\label{kleineri} Ist $0<a<b$, dann ist $0<\frac1b<\frac1a$.
\item $|ab| = |a||b|$
\item $\left| \frac{a}{b} \right| = \frac{|a|}{|b|}$, falls $b\not=0$
\end{enumerate}
\end{aufg}


\bigskip

\begin{lsg}\mbox{ }
\begin{enumerate}[label=$\mathrm{(\roman*)}$, ref=$\mathrm{\roman*}$]
\item 
\item 
\end{enumerate}
\end{lsg}

\bigskip



\begin{aufg}[4 Punkte]\label{mittel}
Zeigen Sie: Für $m,n\in Z$ mit $m<n$ gilt:
\[ m<\frac{m+n}{2} < n\,.\]
Zur Erinnerung: $2 = 1+1$.
\end{aufg}
 
\bigskip 
 
\begin{lsg}[L\"osung~1, Laura Krafft-Schöning und Nele Hansen]
Für m, n $ \in Z$, 
mit $m < n$ gilt:
\[m<\frac{m+n}{2}<n\] 
\\ Beweis:
\\ $m<n \overset{2.26}{\Rightarrow} m+n < n+n = m+n < 2n \overset{2.26}{\Rightarrow} \frac{m+n}{2}< \frac{1}{2}\cdot 2 \cdot n \overset{2.14}{=} n $ (*)
\\
\\$ m<n\overset{2.26}{\Rightarrow} m+m < n+m = 2m < n+m \overset{2.26}{\Rightarrow} \frac{1}{2}\cdot2m \overset{2.14}{=} m < \frac{m+n}{2} $ (**)
\\
\\ Aus den Gleichungen (*) und (**) folgt: $m<\frac{m+n}{2}<n$
\end{lsg}

\bigskip

\begin{lsg}[L\"osung~2]
	~\\[2ex]
	\begin{tabular}{ll}
		Es gilt: & $m<n \Rightarrow m+n<n+n \Rightarrow m+n<2n \Rightarrow m< \frac{m+n}{2}$ \\
		&\\
	Weiterhin gilt: & $m<n \Rightarrow m+m < n+m \Rightarrow 2m < m+n \Rightarrow m<\frac{m+n}{2} \qed$\\
	\end{tabular}



\end{lsg}


\bigskip


\begin{aufg}[6 Punkte]
Es seien $x,y\in Z$. Zeigen Sie, dass die gr\"o{\ss}te untere Schranke $\inf(\{x,y\})$ von $\{x,y\}$ gerade 
\[
\frac{x+y - |y-x|}{2}
\]
ist und leiten Sie eine entsprechende Formel f\"ur die kleinste obere Schranke her.
\end{aufg}


\bigskip

\begin{lsg}
\end{lsg}


\bigskip


\begin{aufg}[6 Punkte]
Bestimme die gr\"o{\ss}te untere und die kleinste obere Schranke der folgenden Mengen (verstanden als Teilmengen von $Z=\R$):
\begin{enumerate}[label=$\mathrm{(\roman*)}$, ref=$\mathrm{\roman*}$]
\item $X:=\{ \frac{1}{n} \mid n\in \N\}$.
\item $X:=\{ x \in Z \mid x < 1 \}$.
\end{enumerate}
\end{aufg}
 
\bigskip

\begin{lsg}[Dennis Oestmann und Tobias Rauer]\mbox{ }
\begin{enumerate}[label=$\mathrm{(\roman*)}$, ref=$\mathrm{\roman*}$]
\item 
Vermutung: sup$(X)=1$
\begin{proof}
Annahme: $\exists x \in X:x>1$. Daraus folgt:
\begin{align*}
&x&&>1 \\
\iff &\frac{1}{n}&&>1 \\
\iff &1&&>n 
\end{align*}  
Da keine natürliche Zahl $n<1$ existiert, ist die Annahme widersprüchlich und $1$ ist eine obere Schranke.

Es gilt zudem: $1=\frac{1}{1}\in X$. Da daher jede potenzielle kleinere obere Schranke für $X$ kleiner als $1$, und damit kleiner als ein Element von $X$ wäre, ist 1 zudem die kleinste obere Schranke.
\end{proof}

Vermutung: inf$(X)=0$
\begin{proof}
0 ist eine untere Schranke, da es keine $n \in \N$ gibt, die kleiner als 0 sind, womit auch $\frac{1}{n} \geq 0$ gilt.

Annahme: Es gibt eine untere Schranke für $X$, die größer als 0 ist. Diese Schranke sei $w$ mit $w>0$. Da $w$ eine untere Schranke ist, gilt $\forall n\in \N: w \leq \frac{1}{n}$. Da $w>0$ gilt, folgt:
\begin{align*}
&w &&\leq \frac{1}{n} \\
\iff &wn &&\leq 1 \\
\iff &n &&\leq \frac{1}{w} 
\end{align*}
Daraus folgt, dass die Menge der natürlichen Zahlen durch $\frac{1}{w}$ nach oben beschränkt ist.
Da $\N \subseteq \R$, gilt nach dem Vollständigkeitsaxiom, dass $\N$ damit auch ein Supremum haben muss. Dieses sei $S$. Es ergibt sich, dass es ein $m \in \N$ geben muss, dass zwischen $S$ und $S-1$ liegt. (Wäre $m>S$, dann wäre $S$ keine obere Schranke, und wäre $m<S-1$, dann wäre $S-1$ eine kleinere Schranke als $S$; $S$ also kein Supremum).

Es gilt also:
\begin{align*}
&S-1&&<m \\
\iff &S&&<m+1 
\end{align*}
$m+1$ ist aber laut Definition auch eine natürliche Zahl. Damit gibt es ein Element in $\N$, das größer ist als das Supremum. Das ist ein Widerspruch und die Annahme muss falsch sein. Die natürlichen Zahlen sind somit nach oben unbeschränkt, und es gibt keine untere Schranke für $X$, die größer als 0 ist. 
\end{proof}
\item 
Vermutung: sup$(X)=1$
\begin{proof}
1 ist eine obere Schranke, da sich die Menge durch $x<1$ definiert.

Annahme:Es gibt eine obere Schranke $w$ von $X$, die kleiner als 1 ist. Dann existiert ein beliebiges, aber festes $v\in \R$ so, dass gilt: $v>0 \land w=1-v<1$. 

Nun existiert für jedes festes $v$ aber ein $1-\frac{v}{2}$. Dieses liegt in $\R$ und ist durch $v>0$ auch kleiner als 1. Daraus folgt, dass $1-\frac{v}{2}$ in $X$ liegt. Es gilt jedoch: $1-\frac{v}{2}>1-v$. Daher ist $1-v$ keine obere Schranke und die Annahme ist widersprüchlich. 1 ist also tatsächlich die kleinste obere Schranke von $X$.
\end{proof}

Vermutung: $X$ ist nach unten unbeschränkt.
\begin{proof}
$X$ ist eine Teilmenge von $\R$. Aus dem Vollständigkeitsaxiom folgt daher, dass wenn $X$ nach unten beschränkt ist, auch ein Infimum existiert. 

Annahme: Es existiert ein solches Infimum, dieses sei $w$.

Man betrachte $w+1$. Für $w+1$ gibt es eine Zahl $z \in X$, für die gilt: $z<w+1$. (Ansonsten wäre $w+1$ eine untere Schranke, die größer ist als $w$, und $w$ damit kein Infimum). Daraus folgt:
\begin{align*}
&z&&<w+1 \\
\iff &z-1&&<w
\end{align*} 
Aus $z\in X$ und damit $z<1$ folgt aber, dass $z-1<1$ und damit $(z-1) \in X$. Da somit ein Element in $X$ existiert, das kleiner als $w$ ist, ist $w$ kein Infimum und die Annahme ist widersprüchlich. $X$ ist also tatsächlich nach unten unbeschränkt.
\end{proof}
\end{enumerate}
\end{lsg}



\section*{3.2}
Laura Krafft-Schöning und Nele Hansen
\\

Für m, n $ \in Z$, 
mit $m < n$ gilt:
\[m<\frac{m+n}{2}<n\] 
\\ Beweis:
\\ $m<n \overset{2.26}{\Rightarrow} m+n < n+n = m+n < 2n \overset{2.26}{\Rightarrow} \frac{m+n}{2}< \frac{1}{2}\cdot 2 \cdot n \overset{2.14}{=} n $ (*)
\\
\\$ m<n\overset{2.26}{\Rightarrow} m+m < n+m = 2m < n+m \overset{2.26}{\Rightarrow} \frac{1}{2}\cdot2m \overset{2.14}{=} m < \frac{m+n}{2} $ (**)
\\
\\ Aus den Gleichungen (*) und (**) folgt: $m<\frac{m+n}{2}<n$



\newpage
\section{Blatt}

\begin{aufg}[6 Punkte]\mbox{ }
\begin{enumerate}[label=$\mathrm{(\roman*)}$, ref=$\mathrm{\roman*}$]
\item Berechnen Sie $\left(-1+i \right)^{10}$ und $\left(-1-i \right)^{10}$. \\
(Hinweis: F\"ur die L\"osung ben\"otigen Sie jeweils maximal zwei Zeilen.)
\item Zeichnen Sie in der komplexen Ebene die Menge
\[
 A \coloneqq \{ z\in\C \mid 2|z|^2 + \Rea z \geq 0\}\,.
\]
\end{enumerate}
\end{aufg}

\bigskip

\begin{lsg}\mbox{ }
\begin{enumerate}[label=$\mathrm{(\roman*)}$, ref=$\mathrm{\roman*}$]
\item Es ist
%
\item 
\end{enumerate}
\end{lsg}

\bigskip

\begin{aufg}[6 Punkte]
Untersuchen Sie folgende Mengen auf Supremum, Maximum, Infimum und Minimum:
\begin{enumerate}[label=$\mathrm{(\roman*)}$, ref=$\mathrm{\roman*}$]
\item $M\coloneqq \left\{ \frac{x}{x-1} : x\in (1,\infty) \right\}$.
\item $K\coloneqq \left\{ \frac{a+b^2}{ab^2} : a\in\N\,,\ b\in\Z, b\not=0\right\}$.
\end{enumerate}
Hierbei ist 
\[
 (1,\infty) \coloneqq \{ x\in \R \mid 1<x \}\,.
\]
\end{aufg}
 
\bigskip

\begin{lsg}
\begin{enumerate}[label=$\mathrm{(\roman*)}$, ref=$\mathrm{\roman*}$]
\item 
\end{enumerate}
\end{lsg}


\bigskip


\begin{aufg}[6 Punkte]
Seien $A$ und $B$ nichtleere Teilmengen von $\R$ und $r\in\R$, $r\leq 0$. Wir definieren 
\[
rA = \lbrace ra \mid a\in A\rbrace
\]
sowie
\[
A+B = \lbrace a+b \mid a\in A, \; b\in B \rbrace\,.
\]
Zeigen Sie unter geeigneten Bedingungen (welche?):
\begin{enumerate}[label=$\mathrm{(\roman*)}$, ref=$\mathrm{\roman*}$]
\item $\sup(rA) = r\inf A$.
\item $\sup(A+B) = \sup A + \sup B$.
\end{enumerate}
\end{aufg}


\bigskip

\begin{lsg}
\begin{enumerate}[label=$\mathrm{(\roman*)}$, ref=$\mathrm{\roman*}$]
\item 
\end{enumerate}
\end{lsg}


\bigskip


\begin{aufg}[6 Punkte] \"Uberpr\"ufen Sie, f\"ur welche $n\in\N$ die folgenden Aussagen jeweils gelten bzw.\@ nicht gelten. Nutzen Sie vollst\"andige Induktion f\"ur den Beweis.
\begin{enumerate}[label=$\mathrm{(\roman*)}$, ref=$\mathrm{\roman*}$]
\item $n!>2^n$.
\item Die Zahl $n^3+2n$ ist durch $3$ teilbar.
\item $2^n > n^3$.
\end{enumerate}
\end{aufg}
 
\bigskip

\begin{lsg}\mbox{ }
\begin{enumerate}[label=$\mathrm{(\roman*)}$, ref=$\mathrm{\roman*}$]
\item 
\end{enumerate}
\end{lsg}

\newpage
\section{Blatt}

\begin{aufg}[8 Punkte] 
Die \textit{Binomialkoeffizienten} ${x\choose k}$ werden f\"ur alle~$x\in \R$ und alle~$ k\in \N_0$ rekursiv definiert durch 
\[
{ x\choose 0 } \coloneqq 1 \quad\text{und} \quad {x\choose k+1} \coloneqq {x\choose k }\cdot\frac{x-k}{k+1} \quad\text{f\"ur} \quad k\geq 0\,.
\]
\begin{enumerate}[label=$\mathrm{(\roman*)}$, ref=$\mathrm{\roman*}$]
\item Berechnen Sie ${-2/3 \choose 3}$.
\item Zeigen Sie, dass f\"ur alle~$n, k\in \N_0$ mit $n\geq k$ gilt
\[
{ n \choose k } = \frac{n!}{k!(n-k)!}\,. 
\]
\item Beweisen Sie folgende Identit\"at f\"ur alle~$n, k\in \N$ mit $1\leq k\leq n$:
\[
{ n \choose k-1 } + { n \choose k } = { n+1 \choose k }\,.
\]
\item Beweisen Sie den \textit{binomischen Lehrsatz}: F\"ur alle~$a,b\in \C$ und f\"ur alle~$n\in \N$ gilt
\[
(a+b)^{n} = \sum_{k=0}^{n} { n \choose k }a^{k}b^{n-k}\,.  
\]
\end{enumerate}
\end{aufg}


\bigskip

\begin{lsg}\mbox{ }
\begin{enumerate}[label=$\mathrm{(\roman*)}$, ref=$\mathrm{\roman*}$]
\item 
\end{enumerate}
\end{lsg}

\bigskip



\begin{aufg}[4 Punkte]
Für alle $n\in \N$ mit $n\ge2$ und alle $x\in\R$ mit $x>-1$ und $x\neq0$ gilt 
\[
(1+x)^n> 1+n\cdot x\,.
\]
Warum ist die Voraussetzung $n\ge2$ erforderlich?
\end{aufg}
 

\bigskip

\begin{lsg}
\end{lsg}


\bigskip


\begin{aufg}[6 Punkte]
\"Uberpr\"ufen Sie, ob die Folge $ (a_{n})_{n\in \mathbb{N}} $ konvergiert und bestimmen Sie gegebenenfalls $\lim_{n\to\infty} a_{n} $ f\"ur 
\begin{enumerate}[label=$\mathrm{(\roman*)}$, ref=$\mathrm{\roman*}$]
\setlength{\itemsep}{2pt}
\item $a_{n} = \frac{4n + 5}{n^{3}+7}$
\item $a_{n} = \sqrt{n+1} - \sqrt{n}$
\item $a_{n} = \frac{\sqrt{n}}{\sqrt[3]{n}+2}$
\item $a_{n} = \frac{(-1)^{n}n + \sqrt{n}}{n+1}$.
\end{enumerate}
\end{aufg}


\bigskip

\begin{lsg}\mbox{ }
\begin{enumerate}[label=$\mathrm{(\roman*)}$, ref=$\mathrm{\roman*}$]
\item
\end{enumerate}
\end{lsg}

\bigskip


\begin{aufg}[6 Punkte]
Beweisen Sie Satz~3.14.
\end{aufg}
 
\bigskip

\begin{lsg}
\end{lsg}


\newpage
% \section{Blatt}

\begin{aufg}[6 Punkte] 
Eine Schnecke kriecht mit einer konstanten Geschwindigkeit von $10$ cm pro Stunden auf einem unendlich elastischem Gummiband entlang, das zum Zeitpunkt~$t=0$ einen Meter lang ist. Die Schnecke startet zum Zeitpunkt~$t=0$ an einem Ende des Gummibandes und kriecht in Richtung des anderen Endes. Nach Ende jeder vollen Stunde kommt ein kleiner Teufel und zieht das Gummiband gleichm\"a{\ss}ig um einen Meter l\"anger. Entwickeln Sie eine Folge, die beschreibt, welchen Anteil des Weges die Schnecke nach $n$ Stunden zur\"uckgelegt hat. Untersuchen Sie, ob die Schnecke das andere Ende des Gummibandes erreicht.
\end{aufg}

\bigskip

\begin{lsg}
\end{lsg}

\bigskip

\begin{aufg}[6 Punkte]
Es sei $(a_n)_n$ eine Folge in~$\R^+_0$, die gegen $a\geq 0$ konvergiert. Zeigen  Sie:
\begin{enumerate}[label=$\mathrm{(\roman*)}$, ref=$\mathrm{\roman*}$]
 \item Die Folge $(\sqrt{a_n})_n$ konvergiert gegen $\sqrt{a}$.
 \item Ist $a\not=0$, dann ist $\lim_{n\to\infty} \sqrt[n]{a_n} = 1$. Was passiert f\"ur $a=0$?
\end{enumerate}
\end{aufg}
 
\bigskip

\begin{lsg}
\begin{enumerate}[label=$\mathrm{(\roman*)}$, ref=$\mathrm{\roman*}$]
\item 
\end{enumerate}
\end{lsg}


\bigskip


\begin{aufg}[6 Punkte]
Es sei $(a_n)_n$ eine Folge in~$\R^+$. Zeigen Sie:
\[
 \lim_{n\to\infty} \frac{1}{\sum_{k=1}^n (a_k + \frac{1}{a_k})} = 0\,.
\]
\end{aufg}

\bigskip

\begin{lsg}  
\end{lsg}

\bigskip


\begin{aufg}[6 Punkte]
Zeigen Sie: 
\begin{enumerate}[label=$\mathrm{(\roman*)}$, ref=$\mathrm{\roman*}$]
\item Die Folge $((1+\frac{1}{n})^n)_{n\in\N}$ ist monoton steigend.
\item Die Folge $((1+\frac{1}{n})^{n+1})_{n\in\N}$ ist monoton fallend.
\item Beide Folgen sind konvergent. 
\end{enumerate}
\end{aufg}
 
\bigskip

\begin{lsg}\mbox{ }
\begin{enumerate}[label=$\mathrm{(\roman*)}$, ref=$\mathrm{\roman*}$]
\item 
\end{enumerate}
\end{lsg}

\bigskip

\begin{aufg}[Bonusaufgabe, 2 Punkte]
 Beweisen Sie Bemerkung~3.12(i)
\end{aufg}

\bigskip

\begin{lsg}
\end{lsg}
 

% \newpage
% \section{Blatt}

\begin{aufg}[6 Punkte]
Beweisen Sie Satz~3.42.
\end{aufg}

\bigskip

\begin{lsg}
\end{lsg}

\bigskip


\begin{aufg}[6 Punkte]
Bestimmen Sie alle H\"aufungswerte der Folge~$(x_n)_{n\in\N}$ mit 
\[
 x_n \coloneqq (-1)^{\lfloor \frac{n}{2}\rfloor} \left( 7 + (-1)^n\left(1+\frac1n\right)^{n+1} \right)\,. 
\]
Bestimmen Sie au{\ss}erdem $\limsup x_n$ und $\liminf x_n$.
\end{aufg}

\bigskip

\begin{lsg}
\end{lsg}

\bigskip

\begin{aufg}[6 Punkte]
Es sei $(a_n)_{n\in\N}$ eine beschr\"ankte Folge in~$\R$. Wir definieren die Folge~$(b_n)_{n\in\N}$ der \emph{arithmetischen Mittel} durch
\[
 b_n \coloneqq \frac1n\sum_{k=1}^n a_k\,.
\]
Zeigen Sie:
\begin{enumerate}[label=$\mathrm{(\roman*)}$, ref=$\mathrm{\roman*}$]
\item Es gilt $\liminf a_n \leq \liminf b_n \leq \limsup b_n \leq \limsup a_n$.
\item Wenn $(a_n)_n$ konvergiert, dann konvergiert auch $(b_n)$. Was ist dann der Grenzwert von $(b_n)$?
\end{enumerate}
\end{aufg}
 
\bigskip

\begin{lsg}\mbox{ }
\begin{enumerate}[label=$\mathrm{(\roman*)}$, ref=$\mathrm{\roman*}$]
\item 
\end{enumerate}
\end{lsg}

\bigskip

\begin{aufg}[6 Punkte]
Es seien $\alpha$ und $x_1$ reelle \textbf{positive} Zahlen. Weiterhin sei die Folge~$(x_n)_{n\in\N}$ definiert durch 
\[
 x_{n+1} \coloneqq \frac12\left(x_n + \frac{\alpha}{x_n}\right) \quad\text{f\"ur $n\in\N$.}
\]
Zeigen Sie, dass die Folge $(x_n)_n$ konvergiert und bestimmen Sie ihren Grenzwert.
\end{aufg}

\bigskip

\begin{lsg}  
\end{lsg}




% \newpage
% \section{Blatt}

\begin{aufg}[6 Punkte]
\glqq Ist schon Weihnachten \ldots\grqq, denkt sich der Grinch, der nichts mehr hasst als das Fest der Harmonie. Aus Rache will er aus der harmonischen Reihe alle Summanden, die die Ziffer Null enthalten, streichen und hofft, damit die Welt in Chaos zu st\"urzen. Sein Vorhaben kann nur aufgehalten werden, wenn die so entstandene \emph{unharmonische} Reihe
\[
 \frac{1}{1} + \frac{1}{2} + \frac{1}{3} + \ldots + \frac{1}{9} + \frac{1}{11} + \ldots + \frac{1}{19} + \frac{1}{21} + \ldots + \frac{1}{29} + \frac{1}{31} + \ldots
\]
konvergiert. Konvergiert oder divergiert die Reihe?
\end{aufg}

\bigskip

\begin{lsg}[Carlotta Hohaus, Gina Walter]
Wir teilen die \emph{unharmonische} Reihe auf und betrachten alle Summanden, die die gleiche Stellenanzahl im Nenner haben, zusammen.
	Betrachtet man die Nenner mit \emph{n} Stellen, gibt es genau $9^{n}$ von diesen, da an jeder der \emph{n} Stellen genau $9$ Ziffern stehen können.
	Jeder der n-stelligen Summanden ist nun auf jeden Fall kleiner $\frac{1}{10^{n-1}}$. Somit gilt für die \emph{unharmonische} Reihe folgendes:
	\begin{align*}
		U&<\sum\limits_{n=1}^\infty 9^{n}\frac{1}{10^{n-1}}\\
		&=9\sum\limits_{n=1}^\infty 9^{n-1}\frac{1}{10^{n-1}}\\
		&=9\sum\limits_{n=1}^\infty (\frac{9}{10})^{n-1}\\
		&=9\sum\limits_{n=0}^\infty (\frac{9}{10})^{n}
	 \end{align*}
	Somit haben wir eine geometrische Reihe produziert. Die geometrische Reihe konvergiert, wenn die Basis kleiner als $1$ ist, was $\frac{9}{10}$ erfüllt.
	Diese geometrische Reihe konvergiert gegen $\frac{1}{1-\frac{9}{10}}=\frac{1}{\frac{1}{10}}=10$ und somit ist
	\begin{align*}
		U<9*10=90
	\end{align*}
	Da U zusätzlich streng monoton wachsend, konvergiert die \emph{unharmonische} Reihe und somit wurde der Grinch bei seinem Vorhaben, die Welt ins Chaos zu st\"urzen, aufgehalten. 
\end{lsg}


\bigskip


\begin{aufg}[6 Punkte]
\"Uberpr\"ufen Sie, ob die folgenden Reihen konvergieren oder divergieren.
\begin{enumerate}[label=$\mathrm{(\roman*)}$, ref=$\mathrm{\roman*}$]
\item $\sum\limits_{k=1}^\infty \frac{1}{k^2}$
\item $\sum\limits_{n=1}^\infty \frac{n^4}{2^n}$
\item $\sum\limits_{p=5}^\infty \binom{p+2}{p}^{-\frac1p}$
\item $\sum\limits_{n=2}^\infty \frac1n \left( \sqrt[n]{n} - \sqrt[n+1]{n+1} \right)$
\item $\sum\limits_{q=1}^\infty (-1)^{q+1} \frac{\sqrt[q]{q}}{q}$
\item $\sum\limits_{n=100}^\infty \frac{1}{\sqrt{n!}}$
\end{enumerate}
\end{aufg}

\bigskip

\begin{lsg}\mbox{ }
\begin{enumerate}[label=$\mathrm{(\roman*)}$, ref=$\mathrm{\roman*}$]
\item 
\end{enumerate}
\end{lsg}

\bigskip

\begin{aufg}[6 Punkte]
Gegeben sei die Reihe $\sum\limits_{n=0}^\infty 2^{(-1)^n-n}$. Zeigen Sie:
\begin{enumerate}[label=$\mathrm{(\roman*)}$, ref=$\mathrm{\roman*}$]
\item Die Reihe konvergiert; geben Sie eine konvergente Majorante an.
\item Die Konvergenz der Reihe ergibt sich auch mit Hilfe des Wurzelkriteriums.
\item Das Quotientenkriterium gibt keine Information \"uber das Konvergenzverhalten.
\end{enumerate}
\end{aufg}
 
\bigskip

\begin{lsg}\mbox{ }
\begin{enumerate}[label=$\mathrm{(\roman*)}$, ref=$\mathrm{\roman*}$]
\item 
\end{enumerate}
\end{lsg}


\bigskip


\begin{aufg}[6 Punkte]
Es seien $(a_n)_n$ und $(b_n)_n$ Folgen in~$\R^+$ und es existiere 
\[
 \gamma\coloneqq \lim_{n\to\infty} \frac{a_n}{b_n} > 0\,.
\]
Zeigen Sie: die Reihen $\sum a_n$ und $\sum b_n$ haben dasselbe Konvergenzverhalten.
\end{aufg}


\bigskip

\begin{lsg}  
\end{lsg}


\bigskip

\begin{aufg}[2 Punkte; Sonderaufgabe; wird fortgesetzt, Anleitung lesen; separate Abgabe]\label{aufg:sonder1}
Es sei $M\subseteq\R$, $M\not=\emptyset$ und $x>0$ f\"ur alle $x\in\M$. Wir setzen
\[
 \frac1M \coloneqq \left\{ \frac1x \in \R \colon x\in M \right\}\,.
\]
Zeigen Sie: Ist $\inf M >0$, dann gilt 
\[
 \sup \frac1M = \frac{1}{\inf M}\,.
\]
\end{aufg}

\bigskip

\begin{lsg}
\end{lsg}

\bigskip 

\begin{center}
 {\large\textbf{Untenstehende Anleitung zur Sonderaufgabe beachten!}}
\end{center}

\bigskip


\noindent
\textbf{Anleitung zur Sonderaufgabe:} Mit Aufgabe~\ref{aufg:sonder1} machen wir ein Experiment, das Sie n\"achste Woche erfahren werden. Die Aufgabe wird also fortgesetzt. Was m\"ussen Sie im Moment tun?

Jede Abgabegruppe bearbeitet diese Aufgabe bitte mit h\"ochster Priorit\"at und versucht, eine m\"oglichst gute L\"osung aufzuschreiben. Diese Aufgabe schreiben Sie bitte auf ein separates Blatt oder eine separate Datei und senden Sie an Ihre \"Ubungsleiter*innen in einer separaten Datei. Versehen Sie die Abgabe mit Ihren Namen, aber \textbf{nicht} mit Ihrem Matrikelnummern. 

Es ist wichtig, dass jede Gruppe eine Abgabe einreicht. Wenn Sie die Aufgabe nicht vollst\"andig l\"osen k\"onnen, geben Sie bitte eine Teill\"osung oder Erkl\"arung Ihrer Versuche und Ans\"atze ab.

Am 14.12.\@ erfahren Sie, wie das Experiment weitergeht.


% \newpage
% \section{Blatt}


\begin{aufg}[6 Punkte]
Romeo m\"ochte Julia besuchen, aber zwischen den beiden ist ein $9$ Meter breiter Fluss. Romeo hat beliebig viele Holzbalken der L\"ange~$1$~m (Breite ~$50$ cm, H\"ohe $10$~cm) zur Verf\"ugung, aus denen er versucht will, eine Br\"ucke von der Form wie im Bild unten (Abbildung~\ref{fig:romeo}) zu bauen. Weiterhin hat er ein beliebig langes Seil, das er am oberen Ende der Br\"ucke befestigen will, um sich herunterzulassen. Die Br\"ucke ist stabil, wenn f\"ur jeden benutzten Holzbalken gilt, dass der Schwerpunkt der Brückenkonstruktion auf diesem Holzbalken \"uber diesem Holzbalken liegt. Das Gewicht von Romeo und dem Seil d\"urfen wir vernachl\"assigen, ebenso die m\"oglichen Probleme bei der Anbringung des Seiles. Schafft Romeo es, zu Julia zu kommen? (Genaue Begr\"undung!)
%
\begin{figure}
\includegraphics{romeo.pdf}
\caption{Romeos Br\"ucke}\label{fig:romeo} 
\end{figure}
%
\end{aufg}

\bigskip

\begin{lsg}
\end{lsg}


\bigskip


\begin{aufg}[6 Punkte]
Bestimmen Sie folgende Grenzwerte:
\begin{enumerate}[label=$\mathrm{(\roman*)}$, ref=$\mathrm{\roman*}$]
\setlength{\itemsep}{4pt}
\item $ \lim\limits_{x\to \infty} \dfrac{x^{2021}-17x^{10}}{x^{2022}+x+3} $,
\item $ \lim\limits_{x\to 2} \dfrac{x^{2}-3x+2}{x^{2}-x-2} $,
\item $ \lim\limits_{x\to 2} \left(1-\dfrac{2}{x}\right)^{2} \left( \dfrac{x^{7}-4x}{(x-2)^{2}} \right) $.
\end{enumerate}
\end{aufg}

\bigskip

\begin{lsg}[Arne Meyer, Simon Bruns]
\begin{enumerate}[label=$\mathrm{(\roman*)}$, ref=$\mathrm{\roman*}$]
\setlength{\itemsep}{4pt}
\item\[\lim_{x \to\infty}\frac{x^{2021}-17x^{10}}{x^{2022}+x+3} =\lim_{x \to\infty} \frac{\frac{1}{x}-\frac{17}{x^{2012}}}{1+\frac{1}{x^{2021}}+\frac{3}{x^{2022}}}=0 \]		
\item\[\lim_{x \to 2}\frac{x^{2}-3x+2}{x^{2}-x-2}= \lim_{x \to 2} \frac{(x-2)(x-1)}{(x-2)(x+1)}=\frac{1}{3}\]
\item\[ \lim_{x \to 2}\left(1-\frac{2}{x}\right)^2\left(\frac{x^7-4x}{\left(x-2\right)^2}\right)=\lim_{x \to 2}\left(\frac{x-2}{x}\right)^2    \cdot \frac{x^7-4x}{(x-2)^2}=\lim_{x \to 2} \frac{x^6-4}{x}= 30\] 
\end{enumerate}
\end{lsg}

\bigskip 


\begin{lsg}[Yan Lin, Yunling Yang]\mbox{ }
\begin{enumerate}[label=$\mathrm{(\roman*)}$, ref=$\mathrm{\roman*}$]
\item Die Funktionen auf dem Z\"ahler und Nenner sind Polynome. Deshalb sind 
die gesamte Funktion stetig, gleichfalls bei (ii) und (iii).
\begin{flalign}
& \lim_{x\to \infty} \frac{x^{2021}-17x^{10}}{x^{2022} + x +3} && \nonumber \\
 =& \lim_{x\to \infty} 
\frac{1-\frac{17}{x^{2011}}}{x+\frac{1}{x^{2020}}+\frac{3}{x^{2021}}} && 
\nonumber \\
 =& \lim_{x\to \infty} \frac{1}{x} && \nonumber \\
 =& \: 0 \nonumber
\end{flalign}
%
\item 
\begin{flalign}
 &\lim_{x \to 2} \frac{x^2-3x+2}{x^2-x-2} && \nonumber\\
=& \lim_{x \to 2} \frac{(x-2)(x-1)}{(x-2)(x+1)}  \nonumber\\
=& \lim_{x \to 2} \frac{x-1}{x+1} \nonumber \\
=& \:\frac{1}{3} \nonumber
\end{flalign}
%
\item 
\begin{flalign}
& \lim_{x \to 2} \left( 1 - 
\frac{2}{x}\right)^2\left(\frac{x^7-4x}{(x-2^2)}\right) && \nonumber \\
=& \lim_{x \to 2} \left(\frac{1-\frac{2}{x}}{x-2}\right)^2 \cdot (x^7-4x) && 
\nonumber \\
=& \lim_{x \to 2} \frac{x^7-4x}{x^2} && \nonumber \\
=& \lim_{x \to 2} x^5-\frac{4}{x} && \nonumber \\
=& \: 30 \nonumber
\end{flalign}
\end{enumerate}
\end{lsg}


\bigskip

\begin{aufg}[6 Punkte]
Die Exponentialfunktion $\exp\colon\R\to\R$ ist gegeben durch (siehe Vorlesung)
\[
\exp(x)\coloneqq \sum_{k=0}^{\infty}\frac{x^k}{k!}\,. 
\]
\begin{enumerate}[label=$\mathrm{(\roman*)}$, ref=$\mathrm{\roman*}$]
\item Zeigen Sie, dass $\exp(x)\cdot \exp(y)=\exp(x+y)$ \emph{(Hinweis: Cauchy-Produkt)}.
\item Zeigen Sie, dass die Exponentialfunktion auf ganz $\R$ stetig ist, d.h., f\"ur jedes~$x_0\in\R$ ist der Grenzwert von~$\exp$ in~$x_0$ gerade $\exp(x_0)$.
\end{enumerate}
\end{aufg}
 
\bigskip

\begin{lsg}\mbox{Lennart Koliwer, Tjado Edzards}
\begin{enumerate}[label=$\mathrm{(\roman*)}$, ref=$\mathrm{\roman*}$]
\item
Z.z.:
\begin{align*}
\text{exp}(x) \cdot \text{exp}(y) = \text{exp}(x+y) &\Leftrightarrow \sum_{k=0}^{\infty}\frac{x^k}{k!}\cdot \sum_{k=0}^{\infty}\frac{y^k}{k!} = \sum_{k=0}^{\infty}\frac{(x+y)^k}{k!}
\\
\intertext{Mit dem Cauchy-Produkt ergibt sich aus exp($x$)exp($y$) folgendes:}
\sum_{k=0}^{\infty}\frac{x^k}{k!}\cdot \sum_{k=0}^{\infty}\frac{y^k}{k!} &= \sum_{k=0}^{\infty}\sum_{n=0}^{k}\frac{x^n}{n!}\cdot \frac{y^{k-n}}{(k-n)!}
\\
\intertext{Zur Vereinfachung wird die Summe mit 1 multipliziert.}
&=\sum_{k=0}^{\infty}\sum_{n=0}^{k}\frac{x^n}{n!}\cdot \frac{y^{k-n}}{(k-n)!}\cdot \frac{k!}{k!}
\\
&=\sum_{k=0}^{\infty}\sum_{n=0}^{k}\frac{k!\cdot x^n\cdot y^{k-n}}{n!\cdot(k-n)!}\cdot \frac{1}{k!}
\intertext{Dabei lässt sich der Binomialkoeffizient finden.}
&=\sum_{k=0}^{\infty}\sum_{n=0}^{k}\binom{k}{n} \cdot x^n\cdot y^{k-n} \cdot \frac{1}{k!}
\\
&=\sum_{k=0}^{\infty}\frac{1}{k!}\sum_{n=0}^{k}\binom{k}{n} \cdot x^n\cdot y^{k-n}
\intertext{Cool das haben wir ja schon bewiesen :) Blatt 5 Binomischer Lehrsatz: $\sum_{n=0}^{k}\binom{k}{n} \cdot x^n\cdot y^{k-n} = (x+y)^k$}
&=\sum_{k=0}^{\infty}\frac{1}{k!}\cdot (x+y)^k
\\
\text{exp}(x+y)&=\sum_{k=0}^{\infty}\frac{(x+y)^k}{k!}
\end{align*}
\end{enumerate}
\end{lsg}

\bigskip


\begin{aufg}[6 Punkte]
Die \textit{Riemannsche Zeta-Funktion} wird definiert durch die Reihe 
\[
\zeta(s) = \sum_{k=1}^{\infty} \frac{1}{k^{s}}
\]
f\"ur alle $s\in \R$, $s>1$. Zeigen Sie, dass $\zeta(n) < 2$ f\"ur alle nat\"urlichen Zahlen $n\geq 2$. 

\noindent
\emph{Hinweis:} Betrachten Sie zuerst
\[
\sum_{n=2}^{\infty} \sum_{k=2}^{\infty} \frac{1}{k^{n}}\,.
\]
\end{aufg}


\bigskip

\begin{lsg}[Zehra Ciftci, Sude Cinar, Saveen Kassem]

Z.z.: $\zeta(n) < 2$ f\"ur alle nat\"urlichen Zahlen $n\geq 2$. 


Es gilt nach Beispiel 3.60: 

\[ \displaystyle\sum_{k=1}^{\infty}\frac{1}{k^{2}}\]

ist eine konvergente Majorante für

\[ \displaystyle\sum_{k=1}^{\infty}\frac{1}{k^{m}}\] 

mit natürlichem $m > 2$. Außerdem gilt: $k^m > k^2$  genau dann, wenn  
$$\frac{1}{k^2} < \frac{1}{k^m}$$.

Auch gilt, dass die erste Patrialsumme mit dem Wert 1 gleich ist.

In Übungsblatt 8 wurde bewiesen, dass 

\[ \displaystyle\sum_{k=1}^{\infty}\frac{2}{(k+1)}=2\]

eine konvergente Majorante für 

\[ \displaystyle\sum_{k=1}^{\infty}\frac{1}{k^{2}}\]

und dass das erste Glied der Reihe gleich ist.

Somit entspricht: 


\begin{align*}
 \displaystyle\sum_{k=1}^{\infty}\frac{1}{k^{2}} < 
\displaystyle\sum_{k=1}^{\infty}\frac{2}{(k+1)}=2.
\end{align*}

Nach dem Axiom der Transitivität der Anorndnung gilt: 
\begin{align*}
\displaystyle\sum_{k=1}^{\infty}\frac{1}{k^{m}} 
& < \displaystyle\sum_{k=1}^{\infty}\frac{1}{k^{2}}
\\
& < \displaystyle\sum_{k=1}^{\infty}\frac{2}{(k+1)}=2
\end{align*}

Nun kann man mit der Definition der Riemannschen-Zeta-Funktion ausdrücken, dass 
mit einer weiterhin natürlichen Zahl größer 2 folgendes gilt:
\begin{align*}
 \zeta(m) = \displaystyle\sum_{k=1}^{\infty}\frac{1}{k^{m}}
& <
\zeta(2) = \displaystyle\sum_{k=1}^{\infty}\frac{1}{k^{2}}
\\ 
& <
\displaystyle\sum_{k=1}^{\infty}\frac{2}{(k+1)}=2
\end{align*}

also kurz: $$ \zeta(m) < \zeta(2) < 2 $$.

Somit wurde $\zeta(n) < 2$ f\"ur alle nat\"urlichen Zahlen $n\geq 2$ gezeigt.
\end{lsg}
 
\bigskip

\begin{aufg}[2 Punkte; Sonderaufgabe]
Wenn Sie Aufgabe~8.5 abgegeben haben (womit Sie sich 2 Punkte verdient haben), erhalten Sie von ihrer \"Ubungsleiterin oder Ihrem \"Ubungsleiter die Abgabe einer anderen Gruppe. Korrigieren Sie diese Abgabe und bepunkten Sie sie. Sie haben 6 Punkte zur Verf\"ugung. (Die andere Gruppe erh\"alt nicht die Punkte, die Sie vergeben, sondern die 2 Punkte f\"ur das Abgeben.) Achten Sie beim Korrigieren insbesondere auf Folgendes: 
\begin{itemize}
 \item Alle Stellen, die falsch sind, anmerken und erkl\"aren, was falsch ist.
 \item Alle Stellen, die unverst\"andlich sind, anmerken und erkl\"aren, was unverst\"andlich ist.
\end{itemize}
Kurzum, Sie machen die Korrektur so, wie Sie erwarten, dass Ihre Abgaben korrigiert werden. Ihr Korrekturergebnis soll so sein, dass Sie hinterher erkl\"aren k\"onnen, was falsch und was richtig an der Abgabe ist. Sie geben die Korrektur dann mit Ihrer Abgabe von Blatt~9 am 21.12. in den \"Ubungsgruppen ab.
\end{aufg}


% \newpage
% \section{Blatt}

\begin{aufg}[6 Punkte]
Zeigen Sie:
\begin{enumerate}[label=$\mathrm{(\roman*)}$, ref=$\mathrm{\roman*}$]
\item Die Reihe
\[
 \sum_{n=1}^\infty \ln\left( 1 + \frac{1}{n^p} \right)
\]
divergiert f\"ur $p=1$, aber konvergiert f\"ur $p=2$.
\item Die Funktion 
\[
 f\colon (0,\infty)\to\R\,,\quad x\mapsto x + e^{-x} - C
\]
besitzt f\"ur jedes $C>1$ eine Nullstelle. Was passiert f\"ur $C=1$ und was f\"ur $C<1$?
\item Jedes reelle Polynom vom Grad~$3$ besitzt mindestens eine reelle Nullstelle.
\end{enumerate}
\end{aufg}


\bigskip

\begin{lsg}\mbox{ }
\begin{enumerate}[label=$\mathrm{(\roman*)}$, ref=$\mathrm{\roman*}$]
\item 
\end{enumerate}
\end{lsg}


\bigskip


\begin{aufg}[6 Punkte]
Sei die Funktion $f\colon\R\rightarrow\R$ definiert durch 
\[
f(x)\coloneqq \left| \left\lfloor x+\frac{1}{2} \right\rfloor -x \right|\,.
\]
Zeichnen Sie den Graphen der Funktion $f$ in ein geeignetes Koordinatensystem und zeigen Sie:
\begin{enumerate}[label=$\mathrm{(\roman*)}$, ref=$\mathrm{\roman*}$]
    \item F\"ur alle $x\in\R$ gilt $0\le f(x) \le \frac{1}{2}$.
    \item F\"ur alle $x\in \R$ und $n\in\Z$ gilt $f(x+n)=f(x)$.
    \item Die Funktion $f$ ist stetig.
\end{enumerate}
\end{aufg}

\bigskip

\begin{lsg}
\end{lsg}


\bigskip

\begin{aufg}[6 Punkte]
Berechnen Sie den punktweisen Grenzwert der folgenden Funktionenfolgen~$(f_n)_{n\in\N}$ und entscheiden Sie, ob die Konvergenz gleichm\"a{\ss}ig ist:
\begin{enumerate}[label=$\mathrm{(\roman*)}$, ref=$\mathrm{\roman*}$]
\item $f_n(x) = \begin{cases} 0 & x \leq n \\ x-n & x > n \end{cases}$ auf
${}]-\infty, 196560]$ und auf $\R$.
\item $f_n(x) = \frac{x}{1+(nx)^2}$ auf $\R$.
\end{enumerate}
\end{aufg}
 
\bigskip

\begin{lsg}\mbox{ }
\begin{enumerate}[label=$\mathrm{(\roman*)}$, ref=$\mathrm{\roman*}$]
\item 
\end{enumerate}
\end{lsg}

\bigskip

\begin{aufg}[6 Punkte]\mbox{ }
\begin{enumerate}[label=$\mathrm{(\roman*)}$, ref=$\mathrm{\roman*}$]
\item Beweisen Sie folgende Aussage: Sei $(a_n)_{n\in\N}$ eine positive, monoton fallende Nullfolge. Dann konvergiert die Reihe 
\[
\sum_{n=1}^{\infty}a_n
\]
genau dann, wenn die \emph{verdichtete} Reihe 
\[
\sum_{k=0}^{\infty}2^k a_{2^k}
\]
konvergiert. 
\item Nutzen Sie diese Aussage, um das Konvergenzverhalten der Reihe 
\[
\sum_{k=1}^{\infty}\frac{1}{n^a}
\]
f\"ur $a>0$ zu untersuchen. 
\end{enumerate}
\end{aufg}

\bigskip

\begin{lsg}\mbox{ }
\begin{enumerate}[label=$\mathrm{(\roman*)}$, ref=$\mathrm{\roman*}$]
\item 
\end{enumerate}
\end{lsg}

\bigskip


\begin{aufg}[8 Punkte; Bonusaufgabe]\mbox{ }
\begin{enumerate}[label=$\mathrm{(\roman*)}$, ref=$\mathrm{\roman*}$]
\item (Fr\"uhpusher-Bonus) Laden Sie bis zum 18.01.2022, 12:00 Uhr, im git eine L\"osung zu einer Aufgabe der Bl\"atter~$1$-$9$ hoch, die bislang noch keine L\"osung hat. Damit erf\"ullen Sie auch zugleich einen Teil Ihrer Studienleistung. Wenn Sie das schon gemacht haben, erhalten Sie diese~$4$ Bonuspunkte automatisch.
%
\item F\"ur genau welche $x\in\R$ konvergiert die Reihe
\[ 
\sum_{n=1}^\infty\left(x+\frac{1}{n}\right)^n\,?
\]
\end{enumerate}
\end{aufg}


\bigskip

\begin{lsg}
\end{lsg}


% \newpage
% \section{blatt}

\begin{aufg}[6 Punkte]
Gegeben sei die Funktion $f\colon [0,1]\to\R$ mit 
\[
 f(x) \coloneqq 
 \begin{cases}
  \frac1q & \text{f\"ur~$x>0$ mit $x=\frac{p}{q}$ mit $p,q\in\N$ 
teilerfremd},
  \\
  0 & \text{f\"ur $x$ irrational},
  \\
  1 & \text{f\"ur $x=0$}.
 \end{cases}
\]
Untersuchen Sie, in welchen Punkten $f$ stetig und in welchen Punkten unstetig 
ist.
\end{aufg}

\bigskip


\begin{lsg}
\end{lsg}


\bigskip


\begin{aufg}[6 Punkte] \mbox{ }
\begin{enumerate}[label=$\mathrm{(\roman*)}$, ref=$\mathrm{\roman*}$]
\item Die \textbf{Sinusfunktion/-reihe} ist definiert durch 
\[
 \sin(x) \coloneqq \sum_{k=0}^\infty (-1)^k \frac{x^{2k+1}}{(2k+1)!}\,.
\]
Die \textbf{Kosinusfunktion/-reihe} ist definiert durch 
\[
 \cos(x) \coloneqq \sum_{k=0}^\infty (-1)^k \frac{x^{2k}}{(2k)!}\,.
\]
Zeigen Sie, dass diese beiden Reihen auf jeder beschr\"ankten Teilmenge 
von~$\R$ gleichm\"a{\ss}ig konvergieren und dass beide Funktionen auf ganz~$\R$ 
stetig sind.
%
\item Zeigen Sie, dass die Potenzreihe
\[
 \sum_{k=0}^\infty x^k
\]
auf $(-1,1)$ nicht gleichm\"a{\ss}ig konvergiert, aber auf jeder beschr\"ankten 
Teilmenge von~$(-1,1)$.
\end{enumerate}
\end{aufg}

\bigskip


\begin{lsg}\mbox{ }
\begin{enumerate}[label=$\mathrm{(\roman*)}$, ref=$\mathrm{\roman*}$]
\item 
\end{enumerate}
\end{lsg}


\bigskip


\begin{aufg}[6 Punkte]
Berechnen Sie den Konvergenzradius der folgenden Potenzreihen
\begin{enumerate}[label=$\mathrm{(\roman*)}$, ref=$\mathrm{\roman*}$]
\item $\sum_{k=0}^\infty \frac{k+1}{3^k}x^k$
\item $\sum_{k=10}^\infty \frac{(x-12)^{2k}}{1+\frac{1}{k}}$
\item $\sum_{k=2}^\infty k\cdot (x+3)^k$
\item $\sum_{k=0}^\infty \binom{3k}{k} x^k$.
\end{enumerate}
\end{aufg}
 
\bigskip


\begin{lsg}\mbox{ }
\begin{enumerate}[label=$\mathrm{(\roman*)}$, ref=$\mathrm{\roman*}$]
\item 
\end{enumerate}
\end{lsg}


\bigskip


\begin{aufg}[6 Punkte]
Zeigen Sie, dass folgende (Anf\"ange von) Potenzreihenentwicklungen gelten (auf 
den Definitionsbereich achten!):
\begin{enumerate}[label=$\mathrm{(\roman*)}$, ref=$\mathrm{\roman*}$]
\item $\sin^3 x = x^3 -\frac12x^5 + \frac{13}{120}x^7 + \ldots$ f\"ur alle 
$x\in\R$,
\item $e^{-x}\sin x = x - x^2 + \frac13x^3 - \ldots$ f\"ur alle $x\in\R$,
\item $\frac{e^x \sin x}{\cos^2 x} = x + x^2 + \frac43 x^3 + x^4 + \ldots$ 
f\"ur hinreichend kleines~$|x|$.
\end{enumerate}
Hierbei ist $\sin^3 x \coloneqq (\sin x)^3$, analog f\"ur $\cos^2 x$.
\end{aufg}


\bigskip


\begin{lsg}\mbox{ }
\begin{enumerate}[label=$\mathrm{(\roman*)}$, ref=$\mathrm{\roman*}$]
\item
\end{enumerate}
\end{lsg}

\bigskip


\begin{aufg}[4 Punkte; Bonusaufgabe]
Untersuchen Sie die Funktionenfolge $(f_n)_n$ mit 
\[
 f_n(x) \coloneqq \sqrt[x]{(x^2+2)^2}
\]
hinsichtlich punktweiser und gleichm\"a{\ss}iger Konvergenz auf~$\R$.
\end{aufg}


\bigskip

\begin{lsg}
\end{lsg}

 


\vspace*{.5cm}


\textbf{\LARGE Wichtig:} Dieses ist das letzte \"Ubungsblatt, das in Analysis~1 
bewertet wird. Das n\"achste \"Ubungsblatt ist ein Ferienblatt, das zu Beginn 
von Analysis~2 besprochen wird (zumindest f\"ur Nicht-Lehramt). \"Uberpr\"ufen 
Sie nun folgendes f\"ur die Studienleistung:
\begin{itemize}
\item Haben Sie $50\%$ aller \"Ubungspunkte erreicht? (Die \"Ubungsleiter*innen 
k\"onnen Ihnen dabei helfen.)
\item Haben Sie zwei Aufgaben vorgerechnet?
\item Haben Sie die L\"osung einer Aufgabe ins git gepusht?
\end{itemize}
Wenn Ihnen noch etwas fehlt, versuchen Sie es noch zu schaffen. 

F\"ur die Nicht-Lehr\"amtler: Abgabe der Plenumsmappen bis 05.02.2022 (in 
Ausnahmef\"allen Verl\"angerung bis 11.02.2022). Sie k\"onnen die Plenumsmappen 
mir per Email senden oder auch ins Postfach legen (Fach~90 im MZH, Ebene~1, 
zwischen Senatssaal und Fahrst\"uhlen).

\setlength{\parindent}{0pt}

\end{document}

% \newpage
% \input{Blatt12}
% \newpage
% \input{Blatt13}

\setlength{\parindent}{0pt}

\end{document}
