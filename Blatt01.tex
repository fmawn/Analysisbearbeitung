\section{Blatt}


\begin{aufg}[2 Punkte]\mbox{ }
\begin{enumerate}[label=$\mathrm{(\roman*)}$, ref=$\mathrm{\roman*}$]
\item Lesen Sie im Stud.IP-Forum den Teil zu git und gitlab durch, schauen Sie die verlinkten Videos, lesen Sie (mindestens einen Teil) der Anleitungen.
\item Besorgen Sie sich eine Freischaltung für den gitlab-Server im FB~3. Im Stud.IP-Forum finden Sie eine Anleitung.
\item Lassen Sie sich zum git repository f\"ur die Analysis~1 eintragen. (Die ersten werden von uns eingetragen. Sprechen Sie z.B.\@ Ihre \"Ubungsleiter*innen an. Alle Eingetragenen k\"onnen dann weitere Personen eintragen, sobald diese Teil~(i) erledigt haben. Das geht bei ``Project Information'', dann ``Members''. Dabei Status ``maintainer'' ausw\"ahlen.) 
\end{enumerate}
\end{aufg}

\bigskip

\begin{aufg}[6 Punkte]
Es seien $A$, $B$ und $C$ Mengen. Beweisen Sie: 
\begin{enumerate}[label=$\mathrm{(\roman*)}$, ref=$\mathrm{\roman*}$]
 \item Es gilt 
 \[
  A\cup B = A\cap B \quad\Leftrightarrow\quad  A=B\,.
 \]
 \item Es gilt 
 \[
  A \cap (B \cup C) = (A\cap B) \cup (A\cap C)\,.
 \]
 \item Es gilt 
 \[
  A \cup (B \cap C) = (A\cup B) \cap (A\cup C)\,.
 \]
\end{enumerate}
\end{aufg}
 
\bigskip 

\begin{lsg}
 
\end{lsg} 

\bigskip


\begin{aufg}[6 Punkte]
Sind die folgenden Relationen Funktionen?
\begin{enumerate}[label=$\mathrm{(\roman*)}$, ref=$\mathrm{\roman*}$]
\item $f \coloneqq \{(n,m)\in \N\times\N \mid n=m^2\}$
\item $f \coloneqq \{(x,y)\in \R\times\R \mid x=y^2\}$
\item $f \coloneqq \{(n,m)\in \N\times\N \mid n^2=1+m\}$
\end{enumerate}
F\"ur diese Aufgabe d\"urfen Sie alle bekannten Eigenschaften f\"ur $\N=\{1,2,3,\ldots\}$ und $\R$ verwenden.
\end{aufg}

\bigskip 

\begin{lsg}
 
\end{lsg}

\bigskip

\begin{aufg}[6 Punkte]
Bei einer Bev\"olkerungsbefragung wird eine Frau in ihrem Haus befragt, wer dort wohne. Sie antwortet: \glqq Mein Mann und ich mit unseren drei T\"ochtern.\grqq{} Auf die Frage nach dem Alter der T\"ochter antwortet sie: \glqq Multipliziert man ihr Alter erh\"alt man~$36$. Die Summe ihrer Alter ist unsere Hausnummer.\grqq{} Der Befrager liest die Hausnummer ab, denkt kurz nach und sagt dann: \glqq Mit den Informationen kann man die Alter ihrer T\"ochter nicht bestimmen.\grqq{}. Die Frau antwortet: \glqq Ja, da haben Sie Recht. Dann sage ich Ihnen noch, dass meine \"alteste Tochter gerade in ihrem Zimmer schl\"aft.\grqq{} Der Befrager antwortet: \glqq Dankesch\"on!\grqq{} und geht gl\"ucklich seines Weges. 

Wie alt sind die T\"ochter? (Selbstverst\"andlich mit Begr\"undung.)
\end{aufg}

\bigskip

\begin{lsg}[Rebekka Dederer, Alexander Polle]
Ges.: $t_1,t_2,t_3$ mit $t_x:=\{t_1,t_2,t_3\}$\\
$t_x :=$ Alter von Tochter x \\
Aus der Aufgabenstellung wird klar:\\ $t_1 \le t_2 \le t_3$ \\
Außerdem gilt für $t_1+t_2+t_3=y$ mit $y:=$ Hausnummer \\und\\
$t_1\cdot t_2\cdot t_3 =36$ \\
Die möglichen Kombinationen an $t_x$ Produkten ist in Tabelle 1. dargestellt.
\begin{table}[h]
\centering
\begin{tabular}{lllll}
$t_1$ & $t_2$ & $t_3$ \\
 1 & 1  & 36    \\
 1 & 2 & 18  \\
 1 & 3 & 12   \\
 1 & 4 & 9    \\
 1 & 6 & 6   \\
 2 & 2 & 9    \\
 2 & 3 & 6    \\
 3 & 3 & 4   \\
\end{tabular}
\caption{Alle nach der Bedingung möglichen Produkte}
\end{table}

 \newpage

Die dazugehörige Summe der jeweiligen $t_x$ als Hausnummer ist in Tabelle 2 dargelegt.\\
\begin{table}[h]
\centering
\begin{tabular}{lllll}
$t_1$ & $t_2$ & $t_3$ & $\sum$\\
 1 & 1  & 36  & 38  \\
 1 & 2 & 18  & 21\\
 1 & 3 & 12  & 16\\
 1 & 4 & 9    & 14\\
 1 & 6 & 6  & 13 \\
 2 & 2 & 9   & 13 \\
 2 & 3 & 6   & 11 \\
 3 & 3 & 4   & 10\\
\end{tabular}
\caption{Alle möglichen Hausnummern durch Summe}
\end{table} \\
Da aus dem Text folgt, dass keine eindeutige Summe aus $t_1,t_2,t_3$ besteht, können also nur Kombination $t_{x,1}:=(1,6,6)$ oder $t_{x,2}:=(2,2,9)$ als Ergebnisse in Frage kommen, da diese die Summe 13 teilen. Aufgrund der Information, dass es genau eine älteste Tochter gibt, lässt sich $t_1,t_2 < t_3$ folgern. Dadurch scheidet Kombination $t_{x,2}$ aus, da hier $t_1 < t_2 = t_3$ gilt. Kombination $t_{x,1}$ erfüllt die Voraussetzungen mit $ t_1,t_2<t_3$ weswegen die Alter der Töchter auf $t_1=2, t_2=2, t_3=9$ festgelegt werden können.

\end{lsg}
 

