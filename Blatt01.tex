\section*{Blatt 1}

\setcounter{blatt}{1}

\begin{aufg}[2 Punkte]\mbox{ }
\begin{enumerate}[label=$\mathrm{(\roman*)}$, ref=$\mathrm{\roman*}$]
\item Lesen Sie im Stud.IP-Forum den Teil zu git und gitlab durch, schauen Sie die verlinkten Videos, lesen Sie (mindestens einen Teil) der Anleitungen.
\item Besorgen Sie sich eine Freischaltung für den gitlab-Server im FB~3. Im Stud.IP-Forum finden Sie eine Anleitung.
\item Lassen Sie sich zum git repository f\"ur die Analysis~1 eintragen. (Die ersten werden von uns eingetragen. Sprechen Sie z.B.\@ Ihre \"Ubungsleiter*innen an. Alle Eingetragenen k\"onnen dann weitere Personen eintragen, sobald diese Teil~(i) erledigt haben. Das geht bei ``Project Information'', dann ``Members''. Dabei Status ``maintainer'' ausw\"ahlen.) 
\end{enumerate}
\end{aufg}

\bigskip

\begin{aufg}[6 Punkte]
Es seien $A$, $B$ und $C$ Mengen. Beweisen Sie: 
\begin{enumerate}[label=$\mathrm{(\roman*)}$, ref=$\mathrm{\roman*}$]
 \item Es gilt 
 \[
  A\cup B = A\cap B \quad\Leftrightarrow\quad  A=B\,.
 \]
 \item Es gilt 
 \[
  A \cap (B \cup C) = (A\cap B) \cup (A\cap C)\,.
 \]
 \item Es gilt 
 \[
  A \cup (B \cap C) = (A\cup B) \cap (A\cup C)\,.
 \]
\end{enumerate}
\end{aufg}
 
\bigskip 

\begin{lsg}
 
\end{lsg} 

\bigskip


\begin{aufg}[6 Punkte]
Sind die folgenden Relationen Funktionen?
\begin{enumerate}[label=$\mathrm{(\roman*)}$, ref=$\mathrm{\roman*}$]
\item $f \coloneqq \{(n,m)\in \N\times\N \mid n=m^2\}$
\item $f \coloneqq \{(x,y)\in \R\times\R \mid x=y^2\}$
\item $f \coloneqq \{(n,m)\in \N\times\N \mid n^2=1+m\}$
\end{enumerate}
F\"ur diese Aufgabe d\"urfen Sie alle bekannten Eigenschaften f\"ur $\N=\{1,2,3,\ldots\}$ und $\R$ verwenden.
\end{aufg}

\bigskip 

\begin{lsg}
 
\end{lsg}

\bigskip

\begin{aufg}[6 Punkte]
Bei einer Bev\"olkerungsbefragung wird eine Frau in ihrem Haus befragt, wer dort wohne. Sie antwortet: \glqq Mein Mann und ich mit unseren drei T\"ochtern.\grqq{} Auf die Frage nach dem Alter der T\"ochter antwortet sie: \glqq Multipliziert man ihr Alter erh\"alt man~$36$. Die Summe ihrer Alter ist unsere Hausnummer.\grqq{} Der Befrager liest die Hausnummer ab, denkt kurz nach und sagt dann: \glqq Mit den Informationen kann man die Alter ihrer T\"ochter nicht bestimmen.\grqq{}. Die Frau antwortet: \glqq Ja, da haben Sie Recht. Dann sage ich Ihnen noch, dass meine \"alteste Tochter gerade in ihrem Zimmer schl\"aft.\grqq{} Der Befrager antwortet: \glqq Dankesch\"on!\grqq{} und geht gl\"ucklich seines Weges. 

Wie alt sind die T\"ochter? (Selbstverst\"andlich mit Begr\"undung.)

\end{aufg}
 
\bigskip 

\begin{lsg}
 
\end{lsg}
