\section*{Blatt 2}


\setcounter{blatt}{2}


\begin{aufg}[6 Punkte]
Beweise die folgenden Identit\"aten f\"ur $a,b,c,d \in Z$:
\begin{enumerate}[label=$\mathrm{(\roman*)}$, ref=$\mathrm{\roman*}$]
\item $a+(b-c)=(a+b)-c$
\item $-(b-a)=a-b$
\item $a-(b-c)=(a+c)-b$
\item $-(a+b)=-a-b$
\end{enumerate}
\end{aufg}

\bigskip

\begin{lsg}\mbox{ }
\begin{enumerate}[label=$\mathrm{(\roman*)}$, ref=$\mathrm{\roman*}$]
\item 
%
\item 
%
\item 
%
\item 
\end{enumerate}
\end{lsg}

\bigskip

\begin{aufg}[6 Punkte]
Zeigen Sie: $\left( \{a,b\}, + , \cdot \right)$ mit $a\not=b$ und 
\begin{center}
\begin{tabular}{c|cc}
 $+$ & $a$ & $b$
 \\ \hline
 $a$ & $a$ & $b$
 \\
 $b$ & $b$ & $a$
\end{tabular}
\qquad
\begin{tabular}{c|cc}
 $\cdot$ & $a$ & $b$
 \\ \hline
 $a$ & $a$ & $a$
 \\
 $b$ & $a$ & $b$
\end{tabular}
\end{center}
ist ein K\"orper. Gibt es eine Anordnung (mit Beweis!)?
\end{aufg}
 
\bigskip

\begin{lsg}
\end{lsg}

\bigskip


\begin{aufg}[6 Punkte]
Beweisen Sie die folgenden Identit\"aten f\"ur $a,b,c,d \in
Z$, $b\neq 0$, $d\neq 0$ 
\begin{enumerate}[label=$\mathrm{(\roman*)}$, ref=$\mathrm{\roman*}$]
\item $\frac{a}{b} \cdot \frac{c}{d} = \frac{ac}{bd}$, 
\item $\frac{\frac{a}{b}}{\frac{c}{d}} = \frac{ad}{bc}$ f\"ur $c\not=0$,
\item $c \frac{a}{b} = \frac{ca}{b}$.
\end{enumerate}
\end{aufg}

\bigskip

\begin{lsg}
\end{lsg}

\bigskip


\begin{aufg}[4 Punkte]
\begin{enumerate}[label=$\mathrm{(\roman*)}$, ref=$\mathrm{\roman*}$]
\item Was ist anschaulich der Unterschied zwischen~$\R$ und~$\Q$? (Nutzen Sie Ihr Schulwissen zu~$\R$ und~$\Q$.)
\item Klassische Mousse au chocolat besteht aus 3-4 Zutaten. Ist die Reihenfolge des Zusammenf\"ugens der Zutaten egal oder nicht? In anderen Worten, erf\"ullt die Zubereitung das Assoziativit\"atsaxiom?
\end{enumerate}
\end{aufg}
 
\bigskip

\begin{lsg}\mbox{ }
\begin{enumerate}[label=$\mathrm{(\roman*)}$, ref=$\mathrm{\roman*}$]
\item
%
\item 
\end{enumerate}
\end{lsg}
