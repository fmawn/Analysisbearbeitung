\section*{Blatt 3}

\setcounter{blatt}{3}

\begin{aufg}[6 Punkte]\label{kleiner}
Seien $a,b\in Z$. Beweisen Sie: 
\begin{enumerate}[label=$\mathrm{(\roman*)}$, ref=$\mathrm{\roman*}$]
\item\label{kleineri} Ist $0<a<b$, dann ist $0<\frac1b<\frac1a$.
\item $|ab| = |a||b|$
\item $\left| \frac{a}{b} \right| = \frac{|a|}{|b|}$, falls $b\not=0$
\end{enumerate}
\end{aufg}


\bigskip

\begin{lsg}\mbox{ }
\begin{enumerate}[label=$\mathrm{(\roman*)}$, ref=$\mathrm{\roman*}$]
\item 
\end{enumerate}
\end{lsg}

\bigskip



\begin{aufg}[4 Punkte]\label{mittel}
Zeigen Sie: Für $m,n\in Z$ mit $m<n$ gilt:
\[ m<\frac{m+n}{2} < n\,.\]
Zur Erinnerung: $2 = 1+1$.
\end{aufg}
 

\bigskip

\begin{lsg}
Hier eine L\"osung. Die ist aber falsch.
\end{lsg}


\bigskip


\begin{aufg}[6 Punkte]
Es seien $x,y\in Z$. Zeigen Sie, dass die gr\"o{\ss}te untere Schranke $\inf(\{x,y\})$ von $\{x,y\}$ gerade 
\[
\frac{x+y - |y-x|}{2}
\]
ist und leiten Sie eine entsprechende Formel f\"ur die kleinste obere Schranke her.
\end{aufg}


\bigskip

\begin{lsg}
\end{lsg}


\bigskip


\begin{aufg}[6 Punkte]
Bestimme die gr\"o{\ss}te untere und die kleinste obere Schranke der folgenden Mengen (verstanden als Teilmengen von $Z=\R$):
\begin{enumerate}[label=$\mathrm{(\roman*)}$, ref=$\mathrm{\roman*}$]
\item $X:=\{ \frac{1}{n} \mid n\in \N\}$.
\item $X:=\{ x \in Z \mid x < 1 \}$.
\end{enumerate}
\end{aufg}
 
\bigskip

\begin{lsg}
\begin{enumerate}[label=$\mathrm{(\roman*)}$, ref=$\mathrm{\roman*}$]
\item 
\end{enumerate}
\end{lsg}


