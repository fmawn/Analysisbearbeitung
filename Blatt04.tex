\section{Blatt}

\begin{aufg}[6 Punkte]\mbox{ }
\begin{enumerate}[label=$\mathrm{(\roman*)}$, ref=$\mathrm{\roman*}$]
\item Berechnen Sie $\left(-1+i \right)^{10}$ und $\left(-1-i \right)^{10}$. \\
(Hinweis: F\"ur die L\"osung ben\"otigen Sie jeweils maximal zwei Zeilen.)
\item Zeichnen Sie in der komplexen Ebene die Menge
\[
 A \coloneqq \{ z\in\C \mid 2|z|^2 + \Rea z \geq 0\}\,.
\]
\end{enumerate}
\end{aufg}

\bigskip

\begin{lsg}\mbox{ }
\begin{enumerate}[label=$\mathrm{(\roman*)}$, ref=$\mathrm{\roman*}$]
\item Es ist
%
\item 
\end{enumerate}
\end{lsg}

\bigskip

\begin{aufg}[6 Punkte]
Untersuchen Sie folgende Mengen auf Supremum, Maximum, Infimum und Minimum:
\begin{enumerate}[label=$\mathrm{(\roman*)}$, ref=$\mathrm{\roman*}$]
\item $M\coloneqq \left\{ \frac{x}{x-1} : x\in (1,\infty) \right\}$.
\item $K\coloneqq \left\{ \frac{a+b^2}{ab^2} : a\in\N\,,\ b\in\Z, b\not=0\right\}$.
\end{enumerate}
Hierbei ist 
\[
 (1,\infty) \coloneqq \{ x\in \R \mid 1<x \}\,.
\]
\end{aufg}
 
\bigskip

\begin{lsg}
\begin{enumerate}[label=$\mathrm{(\roman*)}$, ref=$\mathrm{\roman*}$]
\item 
\end{enumerate}
\end{lsg}


\bigskip


\begin{aufg}[6 Punkte]
Seien $A$ und $B$ nichtleere Teilmengen von $\R$ und $r\in\R$, $r\leq 0$. Wir definieren 
\[
rA = \lbrace ra \mid a\in A\rbrace
\]
sowie
\[
A+B = \lbrace a+b \mid a\in A, \; b\in B \rbrace\,.
\]
Zeigen Sie unter geeigneten Bedingungen (welche?):
\begin{enumerate}[label=$\mathrm{(\roman*)}$, ref=$\mathrm{\roman*}$]
\item $\sup(rA) = r\inf A$.
\item $\sup(A+B) = \sup A + \sup B$.
\end{enumerate}
\end{aufg}


\bigskip

\begin{lsg}
\begin{enumerate}[label=$\mathrm{(\roman*)}$, ref=$\mathrm{\roman*}$]
\item 
\end{enumerate}
\end{lsg}


\bigskip


\begin{aufg}[6 Punkte] \"Uberpr\"ufen Sie, f\"ur welche $n\in\N$ die folgenden Aussagen jeweils gelten bzw.\@ nicht gelten. Nutzen Sie vollst\"andige Induktion f\"ur den Beweis.
\begin{enumerate}[label=$\mathrm{(\roman*)}$, ref=$\mathrm{\roman*}$]
\item $n!>2^n$.
\item Die Zahl $n^3+2n$ ist durch $3$ teilbar.
\item $2^n > n^3$.
\end{enumerate}
\end{aufg}
 
\bigskip

\begin{lsg}\mbox{ }
\begin{enumerate}[label=$\mathrm{(\roman*)}$, ref=$\mathrm{\roman*}$]
\item 
\end{enumerate}
\end{lsg}


\bigskip

\begin{aufg}[Bonusaufgabe, 2 Punkte] \mbox{}
\begin{enumerate}[label=$\mathrm{(\roman*)}$, ref=$\mathrm{\roman*}$]
\item Wer ist Don Knuth? (Ganz kurze Antwort reicht, aber nicht direkt aus Wikipedia kopieren.)
\item Lesen Sie §2 und §3 in
\begin{center}
 \url{https://jmlr.csail.mit.edu/reviewing-papers/knuth_mathematical_writing.pdf}
\end{center}
(das sind nur zwei Seiten) und wenden Sie sie an.
\end{enumerate}
\end{aufg}

\bigskip

\begin{lsg}\mbox{ }
\begin{enumerate}[label=$\mathrm{(\roman*)}$, ref=$\mathrm{\roman*}$]
\item Erfinder/Entwickler von TeX.
\end{enumerate}
\end{lsg}

