\section{Blatt}

\begin{aufg}[8 Punkte] 
Die \textit{Binomialkoeffizienten} ${x\choose k}$ werden f\"ur alle~$x\in \R$ und alle~$ k\in \N_0$ rekursiv definiert durch 
\[
{ x\choose 0 } \coloneqq 1 \quad\text{und} \quad {x\choose k+1} \coloneqq {x\choose k }\cdot\frac{x-k}{k+1} \quad\text{f\"ur} \quad k\geq 0\,.
\]
\begin{enumerate}[label=$\mathrm{(\roman*)}$, ref=$\mathrm{\roman*}$]
\item Berechnen Sie ${-2/3 \choose 3}$.
\item Zeigen Sie, dass f\"ur alle~$n, k\in \N_0$ mit $n\geq k$ gilt
\[
{ n \choose k } = \frac{n!}{k!(n-k)!}\,. 
\]
\item Beweisen Sie folgende Identit\"at f\"ur alle~$n, k\in \N$ mit $1\leq k\leq n$:
\[
{ n \choose k-1 } + { n \choose k } = { n+1 \choose k }\,.
\]
\item Beweisen Sie den \textit{binomischen Lehrsatz}: F\"ur alle~$a,b\in \C$ und f\"ur alle~$n\in \N$ gilt
\[
(a+b)^{n} = \sum_{k=0}^{n} { n \choose k }a^{k}b^{n-k}\,.  
\]
\end{enumerate}
\end{aufg}


\bigskip

\begin{lsg}\mbox{ }
\begin{enumerate}[label=$\mathrm{(\roman*)}$, ref=$\mathrm{\roman*}$]
\item 
\end{enumerate}
\end{lsg}

\bigskip



\begin{aufg}[4 Punkte]
Für alle $n\in \N$ mit $n\ge2$ und alle $x\in\R$ mit $x>-1$ und $x\neq0$ gilt 
\[
(1+x)^n> 1+n\cdot x\,.
\]
Warum ist die Voraussetzung $n\ge2$ erforderlich?
\end{aufg}
 

\bigskip

\begin{lsg}
\end{lsg}


\bigskip


\begin{aufg}[6 Punkte]
\"Uberpr\"ufen Sie, ob die Folge $ (a_{n})_{n\in \mathbb{N}} $ konvergiert und bestimmen Sie gegebenenfalls $\lim_{n\to\infty} a_{n} $ f\"ur 
\begin{enumerate}[label=$\mathrm{(\roman*)}$, ref=$\mathrm{\roman*}$]
\setlength{\itemsep}{2pt}
\item $a_{n} = \frac{4n + 5}{n^{3}+7}$
\item $a_{n} = \sqrt{n+1} - \sqrt{n}$
\item $a_{n} = \frac{\sqrt{n}}{\sqrt[3]{n}+2}$
\item $a_{n} = \frac{(-1)^{n}n + \sqrt{n}}{n+1}$.
\end{enumerate}
\end{aufg}


\bigskip

\begin{lsg}\mbox{ }
\begin{enumerate}[label=$\mathrm{(\roman*)}$, ref=$\mathrm{\roman*}$]
\item
\end{enumerate}
\end{lsg}

\bigskip


\begin{aufg}[6 Punkte]
Beweisen Sie Satz~3.14.
\end{aufg}
 
\bigskip

\begin{lsg}
\end{lsg}

