\section{Blatt}

\begin{aufg}[8 Punkte] 
Die \textit{Binomialkoeffizienten} ${x\choose k}$ werden f\"ur alle~$x\in \R$ und alle~$ k\in \N_0$ rekursiv definiert durch 
\[
{ x\choose 0 } \coloneqq 1 \quad\text{und} \quad {x\choose k+1} \coloneqq {x\choose k }\cdot\frac{x-k}{k+1} \quad\text{f\"ur} \quad k\geq 0\,.
\]
\begin{enumerate}[label=$\mathrm{(\roman*)}$, ref=$\mathrm{\roman*}$]
\item Berechnen Sie ${-2/3 \choose 3}$.
\item Zeigen Sie, dass f\"ur alle~$n, k\in \N_0$ mit $n\geq k$ gilt
\[
{ n \choose k } = \frac{n!}{k!(n-k)!}\,. 
\]
\item Beweisen Sie folgende Identit\"at f\"ur alle~$n, k\in \N$ mit $1\leq k\leq n$:
\[
{ n \choose k-1 } + { n \choose k } = { n+1 \choose k }\,.
\]
\item Beweisen Sie den \textit{binomischen Lehrsatz}: F\"ur alle~$a,b\in \C$ und f\"ur alle~$n\in \N$ gilt
\[
(a+b)^{n} = \sum_{k=0}^{n} { n \choose k }a^{k}b^{n-k}\,.  
\]
\end{enumerate}
\end{aufg}


\bigskip

\begin{lsg}\mbox{ }
\begin{enumerate}[label=$\mathrm{(\roman*)}$, ref=$\mathrm{\roman*}$]
\item 
\end{enumerate}
\end{lsg}

\bigskip



\begin{aufg}[4 Punkte]
Zeigen Sie: Für alle $n\in \N$ mit $n\ge2$ und alle $x\in\R$ mit $x>-1$ und $x\neq0$ gilt 
\[
(1+x)^n> 1+n\cdot x\,.
\]
Warum ist die Voraussetzung $n\ge2$ erforderlich?
\end{aufg}
 

\bigskip

\begin{lsg}[Melia Keil, Leena Mädl]\mbox{ }

    \emph{Induktionsanfang}

    Für $n=1$ gilt:
    \begin{align*}
        &&(1+x)^1&>1+1\cdot x\\
        \iff &&1+x&>1+x\\
    \end{align*}
    $\Longrightarrow$ Hier liegt Gleichheit vor, kein echt größer bzw. kleiner.

    Für $n=2$ gilt:
    \begin{align*}
        &&(1+x)^2&>1+2\cdot x\\
        \iff &&1+2x+x^2&>1+2x\\
        \iff &&x^2&>0
    \end{align*}
   Somit ist der Induktionsanfang 2 und nicht 1
   \\ 
    
    \emph{Induktionsvoraussetzung}

    Sei $(1+x)^n> 1+n\cdot x$ für $n=2$ erfüllt.\\
    \\
    \emph{Induktionsschritt von n auf n+1:}
    \begin{align*}
       \text{Z.z.} &&(1+x)^{n+1}&>1+(n+1)\cdot x\\
        &&(1+x)^{n+1}&=(1+x)^n\cdot (1+x)\\
        && &>(1+nx)\cdot(1+x)=1+nx+x+nx^2=1+(n+1)x+nx^2\\
        && &>1+(n+1)x
    \end{align*}
    
    $\Longrightarrow$ Die Aussage gilt für $n\geq2$, da für $n=1$ nur eine Gleichheit und kein echt größer besteht.
    
     $\Longrightarrow$ Aussage per vollständiger Induktion bewiesen.

\end{lsg}


\bigskip


\begin{aufg}[6 Punkte]
\"Uberpr\"ufen Sie, ob die Folge $ (a_{n})_{n\in \mathbb{N}} $ konvergiert und bestimmen Sie gegebenenfalls $\lim_{n\to\infty} a_{n} $ f\"ur 
\begin{enumerate}[label=$\mathrm{(\roman*)}$, ref=$\mathrm{\roman*}$]
\setlength{\itemsep}{2pt}
\item $a_{n} = \frac{4n + 5}{n^{3}+7}$
\item $a_{n} = \sqrt{n+1} - \sqrt{n}$
\item $a_{n} = \frac{\sqrt{n}}{\sqrt[3]{n}+2}$
\item $a_{n} = \frac{(-1)^{n}n + \sqrt{n}}{n+1}$.
\end{enumerate}
\end{aufg}


\bigskip

\begin{lsg}\mbox{ }
\begin{enumerate}[label=$\mathrm{(\roman*)}$, ref=$\mathrm{\roman*}$]
\item
\end{enumerate}
\end{lsg}

\bigskip


\begin{aufg}[6 Punkte]
Beweisen Sie Satz~3.14.
\end{aufg}
 
\bigskip

\begin{lsg}[Maximilian Kuppinger und Lorens Dinklage]
    Seien ($a_n$)$_{n \in \mathbb{N}}$ eine Folge und $a \in \mathbb{C}.$
    
    \textbf{Antwort zu ($i$)}
        Es gilt zu zeigen, dass sei ($\gamma_n$)$_{n \in \mathbb{N}}$ eine Nullfolge und es gelte $$\forall n \in \mathbb{N}: |a_n - a| \le         \gamma_n,$$dann ist $\lim(a_n)=a.$
    
    \begin{proof}
        Betrachten wir zun\"achst, was wir durch unsere Voraussetzung wissen. Da ($\gamma_n$)$_{n \in \mathbb{N}}$ eine Nullfolge ist, gilt nach (\textit{Definition 3.2.}), dass $\lim(\gamma_n)=0$ und somit auch $$\forall \varepsilon \in \mathbb{R}, \varepsilon > 0, \exists n_0 = n_0(\varepsilon) \in \mathbb{N}, n \in \mathbb{N}, \forall n \ge n_0 : |\gamma_n| < \varepsilon.$$
    
        Aus der Voraussetzung geht au{\ss}erdem hervor, dass, da die Betragsfunktion nach (\textit{Definition 2.29.}) nur auf nichtnegative Zahlen abbildet, gilt $0 \le |a_n - a|$ und da wir ebenfalls $|a_n - a| \le \gamma_n$ angenommen haben, also insbesondere $0 \le \gamma_n$, folgt nach (\textit{Definition 2.29.}) $|\gamma_n| = \gamma_n$, also auch $$\forall \varepsilon \in \mathbb{R}, \varepsilon > 0, \exists n_0 = n_0(\varepsilon) \in \mathbb{N}, n \in \mathbb{N}, \forall n \ge n_0 : \gamma_n < \varepsilon.$$Nun k\"onnen wir unsere Aussage hinzuf\"ugen, also $$\forall \varepsilon \in \mathbb{R}, \varepsilon > 0, \exists n_0 = n_0(\varepsilon) \in \mathbb{N}, n \in \mathbb{N}, \forall n \ge n_0 : |a_n - a| \le \gamma_n < \varepsilon,$$was insbesondere bedeutet, dass $$\forall \varepsilon \in \mathbb{R}, \varepsilon > 0, \exists n_0 = n_0(\varepsilon) \in \mathbb{N}, n \in \mathbb{N}, \forall n \ge n_0 : |a_n - a| < \varepsilon,$$was nach (\textit{Definition 3.2.}) bedeutet, dass $a$ Grenzwert von ($a_n$)$_{n \in \mathbb{N}}$ ist, beziehungsweise $\lim$($a_n$)$=a,$ was zu zeigen war.
    \end{proof}
    
    \textbf{Antwort zu ($ii$)}
        Es gilt zu zeigen, dass wenn $\lim$($a_n$)$=a,$ dann $\lim$($|a_n|$)$=|a|.$
    
    \begin{proof}
        Nehmen wir also $\lim$($a_n$)$=a$ an, was wir nach (\textit{Definition 3.2.}) auch wie folgt schreiben k\"onnen $$\forall \varepsilon \in \mathbb{R}, \varepsilon > 0, \exists n_0 = n_0(\varepsilon) \in \mathbb{N}, n \in \mathbb{N}, \forall n \ge n_0 : |a_n - a| < \varepsilon.$$ 
        Nun wollen wir zeigen, dass dann stets auch gilt, dass$$\forall \varepsilon \in \mathbb{R}, \varepsilon > 0, \exists n_0 = n_0(\varepsilon) \in \mathbb{N}, n \in \mathbb{N}, \forall n \ge n_0 : ||a_n| - |a|| < \varepsilon.$$
    
        Dazu nutzen wir nun, dass nach (\textit{Satz 2.8. $ii$}) $-a = +(-a)$ und die gro{\ss}e/erweiterte Dreiecksungleichung, erhalten somit $$\forall \varepsilon \in \mathbb{R}, \varepsilon > 0, \exists n_0 = n_0(\varepsilon) \in \mathbb{N}, n \in \mathbb{N}, \forall n \ge n_0 : |a_n + (-a)| < \varepsilon$$und $$\forall \varepsilon \in \mathbb{R}, \varepsilon > 0, \exists n_0 = n_0(\varepsilon) \in \mathbb{N}, n \in \mathbb{N}, \forall n \ge n_0 : ||a_n|-|-a|| \le |a_n + (-a)| < \varepsilon.$$Nach (\textit{Satz 2.30. $i$}) gilt zudem $|-a| = |a|$ und somit gilt insbesondere $$\forall \varepsilon \in \mathbb{R}, \varepsilon > 0, \exists n_0 = n_0(\varepsilon) \in \mathbb{N}, n \in \mathbb{N}, \forall n \ge n_0 : ||a_n|-|-a|| = ||a_n|-|a|| < \varepsilon,$$oder nach (\textit{Definition 3.2.}) auch $\lim$($|a_n|$)$=|a|,$ was zu beweisen war.
    \end{proof}
\end{lsg}

