\section{Blatt}

\begin{aufg}[6 Punkte] 
Eine Schnecke kriecht mit einer konstanten Geschwindigkeit von $10$ cm pro Stunde auf einem unendlich elastischem Gummiband entlang, das zum Zeitpunkt~$t=0$ einen Meter lang ist. Die Schnecke startet zum Zeitpunkt~$t=0$ an einem Ende des Gummibandes und kriecht in Richtung des anderen Endes. Nach Ende jeder vollen Stunde kommt ein kleiner Teufel und zieht das Gummiband gleichm\"a{\ss}ig um einen Meter l\"anger. Entwickeln Sie eine Folge, die beschreibt, welchen Anteil des Weges die Schnecke nach $n$ Stunden zur\"uckgelegt hat. Untersuchen Sie, ob die Schnecke das andere Ende des Gummibandes erreicht.
\end{aufg}

\bigskip

\begin{lsg} [Oscar Corr und Filco Roedenbeck]
Die Schnecke bewegt sich mit einer konstanten Geschwindigkeit von 0,1m pro Stunde. Dies entspricht in der ersten Stunde $\frac{1}{10}$m. Im Anschluss wird das Gummiband um 1m verlängert, wobei der zurückgelegte Anteil der Strecke gleich bleibt. Im Anschluss bewegt sich die Schnecke um  $\frac{1}{2 \times 10}$m in der 2. Stunde. 
Der zurückgelegte Anteil der Strecke wird durch Folgende Reihe beschrieben: 
$\sum\nolimits_{k=1}^n$ $\frac{1}{n \times 10} $ = zurückgelegter Anteil der Strecke zum Zeitpunkt n.
Es bleibt zu zeigen, dass $\lim_{n \to \infty} \sum\nolimits_{k=1}^n \frac{1}{n \times 10}$ divergiert.\\ 
$\lim_{n \to \infty} \sum\nolimits_{k=1}^n \frac{1}{n \times 10} \\ 
= \lim_{n \to \infty}  \frac{1}{10}\sum\nolimits_{k=1}^n \frac{1}{n} \\
= \lim_{n \to \infty} \frac{1}{10}(1+\frac{1}{2}+\frac{1}{3}+\frac{1}{4}+ \ldots+\frac{1}{n-1}+\frac{1}{n}) \\$
Nach unten Abschätzen $(\frac{1}{3}+\frac{1}{4}\geq 2 \times \frac{1}{4})$: \\
\textgreater \ $ \lim_{n \to \infty} \frac{1}{10} (1+\frac{1}{2}+ (\frac{1}{4}+\frac{1}{4})+(\frac{1}{8}+\frac{1}{8}+\frac{1}{8}+\frac{1}{8})+\ldots + (\frac{n}{2} \times \frac{1}{n})) \\
= \lim_{n \to \infty} \frac{1}{10} (1+\frac{1}{2}+\frac{1}{2}+\frac{1}{2}+\ldots +\frac{1}{2}) \\
= \lim_{n \to \infty} \frac{1}{10} (1+ \sum\nolimits_{k=1}^{n-1} \frac{1}{2})\\
=\lim_{n \to \infty} \frac{1}{10} + \sum\nolimits_{k=1}^{n-1} \frac{1}{20} \\
=\lim_{n \to \infty} \frac{1}{10} + (n-1) \times \frac{1}{20}$
Daraus folgt, dass die Reihe divergiert. Somit muss auch die größere Reihe 
$\lim_{n \to \infty} \sum\nolimits_{k=1}^n \frac{1}{n \times 10}$ divergieren.
Dies zeigt, dass die Schnecke das Ende des Gummibandes erreicht, da sie für eine unbegrenzte Zeit jede Streck überwindet. 
\end{lsg}

\bigskip

\begin{aufg}[6 Punkte]
Es sei $(a_n)_n$ eine Folge in~$\R^+_0$, die gegen $a\geq 0$ konvergiert. Zeigen  Sie:
\begin{enumerate}[label=$\mathrm{(\roman*)}$, ref=$\mathrm{\roman*}$]
 \item Die Folge $(\sqrt{a_n})_n$ konvergiert gegen $\sqrt{a}$.
 \item Ist $a\not=0$, dann ist $\lim_{n\to\infty} \sqrt[n]{a_n} = 1$. Was passiert f\"ur $a=0$?
\end{enumerate}
\end{aufg}
 
\bigskip

\begin{lsg}
\begin{enumerate}[label=$\mathrm{(\roman*)}$, ref=$\mathrm{\roman*}$]
\item 
\end{enumerate}
\end{lsg}


\bigskip


\begin{aufg}[6 Punkte]
Es sei $(a_n)_n$ eine Folge in~$\R^+$. Zeigen Sie:
\[
 \lim_{n\to\infty} \frac{1}{\sum_{k=1}^n (a_k + \frac{1}{a_k})} = 0\,.
\]
\end{aufg}

\bigskip

\begin{lsg}  
\end{lsg}

\bigskip


\begin{aufg}[6 Punkte]
Zeigen Sie: 
\begin{enumerate}[label=$\mathrm{(\roman*)}$, ref=$\mathrm{\roman*}$]
\item Die Folge $((1+\frac{1}{n})^n)_{n\in\N}$ ist monoton steigend.
\item Die Folge $((1+\frac{1}{n})^{n+1})_{n\in\N}$ ist monoton fallend.
\item Beide Folgen sind konvergent. 
\end{enumerate}
\end{aufg}
 
\bigskip

\begin{lsg}\mbox{ }
\begin{enumerate}[label=$\mathrm{(\roman*)}$, ref=$\mathrm{\roman*}$]
\item 
\end{enumerate}
\end{lsg}

\bigskip

\begin{aufg}[Bonusaufgabe, 2 Punkte]
 Beweisen Sie Bemerkung~3.12(i)
\end{aufg}

\bigskip

\begin{lsg}
\end{lsg}
 
