\section{Blatt}

\begin{aufg}[6 Punkte]
Beweisen Sie Satz~3.42.
\end{aufg}

\bigskip

\begin{lsg}
\end{lsg}

\bigskip


\begin{aufg}[6 Punkte]
Bestimmen Sie alle H\"aufungswerte der Folge~$(x_n)_{n\in\N}$ mit 
\[
 x_n \coloneqq (-1)^{\lfloor \frac{n}{2}\rfloor} \left( 7 + (-1)^n\left(1+\frac1n\right)^{n+1} \right)\,. 
\]
Bestimmen Sie au{\ss}erdem $\limsup x_n$ und $\liminf x_n$.
\end{aufg}

\bigskip

\begin{lsg}
\end{lsg}

\bigskip

\begin{aufg}[6 Punkte]
Es sei $(a_n)_{n\in\N}$ eine beschr\"ankte Folge in~$\R$. Wir definieren die Folge~$(b_n)_{n\in\N}$ der \emph{arithmetischen Mittel} durch
\[
 b_n \coloneqq \frac1n\sum_{k=1}^n a_k\,.
\]
Zeigen Sie:
\begin{enumerate}[label=$\mathrm{(\roman*)}$, ref=$\mathrm{\roman*}$]
\item Es gilt $\liminf a_n \leq \liminf b_n \leq \limsup b_n \leq \limsup a_n$.
\item Wenn $(a_n)_n$ konvergiert, dann konvergiert auch $(b_n)$. Was ist dann der Grenzwert von $(b_n)$?
\end{enumerate}
\end{aufg}
 
\bigskip

\begin{lsg}\mbox{ }
\begin{enumerate}[label=$\mathrm{(\roman*)}$, ref=$\mathrm{\roman*}$]
\item 
\end{enumerate}
\end{lsg}

\bigskip

\begin{aufg}[6 Punkte]
Es seien $\alpha$ und $x_1$ reelle \textbf{positive} Zahlen. Weiterhin sei die Folge~$(x_n)_{n\in\N}$ definiert durch 
\[
 x_{n+1} \coloneqq \frac12\left(x_n + \frac{\alpha}{x_n}\right) \quad\text{f\"ur $n\in\N$.}
\]
Zeigen Sie, dass die Folge $(x_n)_n$ konvergiert und bestimmen Sie ihren Grenzwert.
\end{aufg}

\bigskip

\begin{lsg}  
\end{lsg}


\bigskip

\begin{aufg}[Sonderaufgabe; wird fortgesetzt, Anleitung lesen; Abgabe 14.12.2021]\label{aufg:sonder1}
Es sei $M\subseteq\R$, $M\not=\emptyset$ und $x>0$ f\"ur alle $x\in\M$. Wir setzen
\[
 \frac1M \coloneqq \left\{ \frac1x \in \R \colon x\in M \right\}\,.
\]
Zeigen Sie: Ist $\inf M >0$, dann gilt 
\[
 \sup \frac1M = \frac{1}{\inf M}\,.
\]
\end{aufg}

\bigskip

\begin{lsg}
\end{lsg}

\bigskip


\noindent
\textbf{Anleitung zur Sonderaufgabe:} Mit Aufgabe~\ref{aufg:sonder1} machen wir ein Experiment, das Sie n\"achste Woche erfahren werden. Die Aufgabe wird also fortgesetzt. Was m\"ussen Sie im Moment tun?

Jede Abgabegruppe bearbeitet diese Aufgabe bitte mit h\"ochster Priorit\"at und versucht, eine m\"oglichst gute L\"osung aufzuschreiben. Diese Aufgabe schreiben Sie bitte auf ein separates Blatt und per Onlineabgaben in eine separate Datei. Versehen Sie die Abgabe mit Ihren Namen, aber \textbf{nicht} mit Ihrem Matrikelnummern. 

Es ist wichtig, dass jede Gruppe eine Abgabe einreicht. Wenn Sie die Aufgabe nicht vollst\"andig l\"osen k\"onnen, geben Sie bitte eine Teill\"osung oder Erkl\"arung Ihrer Versuche und Ans\"atze ab.

Ihre L\"osungen geben Sie erst am 14.12.\@ bzw.\@ 15.12.\@ in den \"Ubungsgruppen ab und dann erfahren Sie, sie das Experiment weitergeht.
