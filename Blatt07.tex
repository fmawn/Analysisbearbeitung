\section{Blatt}

\begin{aufg}[6 Punkte]
Beweisen Sie Satz~3.42.
\end{aufg}

\bigskip

\begin{lsg} [Melina Feldmann, Justin Krieb]
Sei \(x_{\infty}\) ein Häufungswert von $x$. Nun wählen wir jedes $k\in\mathbb{N}$ als einen Index,
sodass \(n_{k}\)$>$\(n_{k-1}\) mit $($\(x_{n_{k}}\), \(x_{\infty}\)$)$$<$\(\frac{1}{k}\). 
Damit ist $($\(x_{n_{k}}\)$)$ eine Teilfolge von $x$, welche gegen \(x_{\infty}\) konvergiert.

Sei also $($\(x_{n_{k}}\)$)$ eine Teilfolge von $x$, die gegen \(x_{\infty}\) konvergiert. 
Für jedes $\epsilon > 0$ und  $n\in\mathbb{N}$ existiert somit ein $k$ mit \(n_{k}\)$>n$ mit  
$($\(x_{n_{k}}\), \(x_{\infty}\)$)$$< \epsilon$, das heißt \(x_{\infty}\) ist ein Häufungswert von $x$.
\end{lsg}

\bigskip


\begin{aufg}[6 Punkte]
Bestimmen Sie alle H\"aufungswerte der Folge~$(x_n)_{n\in\N}$ mit 
\[
 x_n \coloneqq (-1)^{\lfloor \frac{n}{2}\rfloor} \left( 7 + (-1)^n\left(1+\frac1n\right)^{n+1} \right)\,. 
\]
Bestimmen Sie au{\ss}erdem $\limsup x_n$ und $\liminf x_n$.
\end{aufg}

\bigskip

\begin{lsg}[Siyao Zhang, Alexander Unterberger]
Wir \"uberpr\"ufen $x_n$ auf Teilfolgen:\\
Zuerst betrachten wir alle geraden Zahlen und setzen $n=2k$ mit $k\in \mathbb{N}$\\

\begin{equation}
\begin{aligned}
    x_{2k} &= (-1)^{\lfloor \frac{2k}{2}\rfloor} \left(7+(-1)^{2k}\left(1+\frac{1}{2k}\right)^{2k+1} \right)\\
    &= (-1)^k \left(7+(-1)^{2k}\left(1+\frac{1}{2k}\right)^{2k+1} \right)\\
    &= (-1)^k \left(7+e \right)\\
\end{aligned}
\end{equation}
\\
$\Rightarrow$ Diese Teilfolge hat 2 H\"aufungswerte: $(7+e)$ und $(-7-e)$.\\
\\
Nun betrachten wir alle ungeraden Zahlen und setzen $n=2k+1$ mit $k\in \mathbb{N}$\\
\begin{equation}
\begin{aligned}
    x_{2k+1} &= (-1)^{\lfloor \frac{2k+1}{2}\rfloor} \left(7+(-1)^{2k+1}\left(1+\frac{1}{2k+1}\right)^{2k+2} \right)\\
    &= (-1)^{\lfloor \frac{2}{2}k+\frac{1}{2}\rfloor} \left(7+(-1)\left(1+0\right)^{2k+2} \right)\\
    &= (-1)^k \left(7-1 \right)\\
\end{aligned}
\end{equation}
\\
$\Rightarrow$ Diese Teilfolge hat 2 H\"aufungswerte: $6$ und $-6$.\\
\\
Daraus folgt, dass die Folge insgesamt 4 H\"aufungswerte hat: $6$, $-6$ und $(7+e)$, $(-7-e)$.\\
\\
Nach Satz 3.45 gilt für den gr\"o{\ss}ten bzw. kleinsten H\"aufungswert:\\
\\
$\limsup\limits_{n \to \infty} x_n = \sup\{S_n|n\in\mathbb{N}\}=(7+e)$\\
\\
$\liminf\limits_{n \to \infty} x_n = \inf\{I_n|n\in\mathbb{N}\}=(-7-e)$
\end{lsg}


\bigskip 

\begin{lsg}[Tim Rust, Johanna Constien (zweite Variante)]

$\lfloor \frac{n}{2}\rfloor$ ist die Gaußklammer. Sie bewirkt dass immer zur 
nächst kleineren, ganzen Zahl abgerundet wird z.B. $3,9 \rightarrow 3$.\\

Def.: a ist ein H\"aufungspunkt, falls eine Teilfolge $(a_{n_k})_{k\in\N} 
\xrightarrow[k \rightarrow \infty]{} a$.\\

Durch Ausprobieren sind wir auf vier Teilfolgen gekommen, die alle Zahlen der 
Folge abdeckt.\\

1. Teilfolge:\\
\[
\begin{aligned}
(a_{4k})_{k\in\N}
&=(-1)^{\lfloor \frac{4k}{2}\rfloor} \left( 7 + 
(-1)^{4k}\left(1+\frac{1}{4k}\right)^{4k+1} \right)\\
&=7+\left(1+\frac{1}{4k}\right)^{4k+1}\\
\\
&\lim \limits_{k \to \infty} 7+\left(1+\frac{1}{4k}\right)^{4k+1}\\
&=\lim \limits_{k \to \infty} 7 + \lim \limits_{k \to 
\infty}\left(1+\frac{1}{4k}\right)^{4k+1}\\
&=7+e
\end{aligned}
\]\\

2. Teilfolge:\\
\[
\begin{aligned}
(a_{4k+1})_{k\in\N}
&=(-1)^{\lfloor \frac{4k+1}{2}\rfloor} \left( 7 + 
(-1)^{4k+1}\left(1+\frac{1}{4k+1}\right)^{4k+1+1} \right)\\
&=7+(-1)\left(1+\frac{1}{4k+1}\right)^{4k+2}\\
&=7-\left(1+\frac{1}{4k+1}\right)^{4k+2}\\
\\
&\lim \limits_{k \to \infty} 7-\left(1+\frac{1}{4k+1}\right)^{4k+2}\\
&=\lim \limits_{k \to \infty} 7 - \lim \limits_{k \to 
\infty}\left(1+\frac{1}{4k+1}\right)^{4k+2}\\
&=7-e
\end{aligned}
\]
\\
3. Teilfolge:\\
\[
\begin{aligned}
(a_{4k-1})_{k\in\N}
&=(-1)^{\lfloor \frac{4k-1}{2}\rfloor} \left( 7 + 
(-1)^{4k-1}\left(1+\frac{1}{4k-1}\right)^{4k+1-1} \right)\\
&=-\left(7-\left(1+\frac{1}{4k-1}\right)^{4k}\right)\\
\\
&\lim \limits_{k \to \infty} 
-\left(7-\left(1+\frac{1}{4k-1}\right)^{4k}\right)\\
&=\lim \limits_{k \to \infty} -7+ \lim \limits_{k \to 
\infty}\left(1+\frac{1}{4k-1}\right)^{4k}\\
&=-7+e
\end{aligned}
\]\\
4. Teilfolge:\\
\[
\begin{aligned}
(a_{4k-2})_{k\in\N}
&=(-1)^{\lfloor \frac{4k-2}{2}\rfloor} \left( 7 + 
(-1)^{4k-2}\left(1+\frac{1}{4k-2}\right)^{4k+1-2} \right)\\
&=-\left(7+\left(1+\frac{1}{4k-2}\right)^{4k-1}\right)\\
\\
&\lim \limits_{k \to \infty} 
-\left(7+\left(1+\frac{1}{4k-2}\right)^{4k-1}\right)\\
&=\lim \limits_{k \to \infty} -7- \lim \limits_{k \to 
\infty}\left(1+\frac{1}{4k-2}\right)^{4k-1}\\
&=-7-e
\end{aligned}
\]\\

Die H\"aufungswerte der Folge~$(x_n)_{n\in\N}$ sind $-7-e$, $e-7$, $7-e$ und 
$7+e$.\\

Der $\limsup x_n$ ist der gr\"oßte H\"aufungswerte: 
\[
\limsup x_n= 7+e
\]\\
Der $\liminf x_n$ ist der kleinste H\"aufungswerte: 
\[
\liminf x_n=-7-e
\]\\
\end{lsg}


\bigskip

\begin{aufg}[6 Punkte]
Es sei $(a_n)_{n\in\N}$ eine beschr\"ankte Folge in~$\R$. Wir definieren die Folge~$(b_n)_{n\in\N}$ der \emph{arithmetischen Mittel} durch
\[
 b_n \coloneqq \frac1n\sum_{k=1}^n a_k\,.
\]
Zeigen Sie:
\begin{enumerate}[label=$\mathrm{(\roman*)}$, ref=$\mathrm{\roman*}$]
\item Es gilt $\liminf a_n \leq \liminf b_n \leq \limsup b_n \leq \limsup a_n$.
\item Wenn $(a_n)_n$ konvergiert, dann konvergiert auch $(b_n)$. Was ist dann der Grenzwert von $(b_n)$?
\end{enumerate}
\end{aufg}
 
\bigskip

\begin{lsg}\mbox{ }
\begin{enumerate}[label=$\mathrm{(\roman*)}$, ref=$\mathrm{\roman*}$]
\item 
\end{enumerate}
\end{lsg}

\bigskip

\begin{aufg}[6 Punkte]
Es seien $\alpha$ und $x_1$ reelle \textbf{positive} Zahlen. Weiterhin sei die Folge~$(x_n)_{n\in\N}$ definiert durch 
\[
 x_{n+1} \coloneqq \frac12\left(x_n + \frac{\alpha}{x_n}\right) \quad\text{f\"ur $n\in\N$.}
\]
Zeigen Sie, dass die Folge $(x_n)_n$ konvergiert und bestimmen Sie ihren Grenzwert.
\end{aufg}

\bigskip

\begin{lsg}  

Vorgehensweise:
\begin{enumerate}
    \item Zeige, dass $x_2 > \sqrt{\alpha}$ für $x_1 \neq \sqrt{\alpha}$
    \item Zeige, dass $x_n > \sqrt{\alpha}$ für $n \geq 2$
    \item Zeige, dass $x_n$ monoton fällt für $n \geq 2$
    \item $\lim x_n = \lim x_{n+1} = \sqrt{\alpha}$ ist Grenzwert
\end{enumerate}

Für $x_1$ gibt es die drei Möglichkeiten $x_1 = \sqrt{\alpha}$, $x_1 < \sqrt{\alpha}$, $x_1 > \sqrt{\alpha}$, aus denen jeweils $x_2 \geq \sqrt{\alpha}$ folgt:

\noindent
\underline{Fall 1:} 
\begin{align*}
x_1 &= \sqrt{\alpha} \\
\Rightarrow x_2 &= \frac{1}{2} \left( \sqrt{\alpha} + \frac{\alpha}{\sqrt{\alpha}} \right) = \sqrt{\alpha}
\end{align*}
 also ist für $x_1 = \sqrt{\alpha}$ $x_n$ eine konstante Folge mit $x_n = \sqrt{\alpha}$.

\noindent
\underline{Fall 2:} $x_1 < \sqrt{\alpha} \Rightarrow x_1 = \sqrt{\alpha} - h_1$, wobei $h_1 > 0$ und $h_1 < \sqrt\alpha$
\begin{align*}
\Rightarrow x_2 &= \frac{1}{2} \left( \sqrt{\alpha} - h_1 + \frac{\alpha}{\sqrt{\alpha} - h_1} \right) \\
\left(\text{mit} \frac{1}{\sqrt{\alpha} - h_1} > \frac{1}{\sqrt{\alpha}}\right)\quad \Rightarrow x_2 &> \frac{1}{2} \left( \sqrt{\alpha} - h_1 + \frac{\alpha}{\sqrt{\alpha}} \right) \\
(\text{mit} -h_1 < 0)\quad \Rightarrow x_2 &> \frac{1}{2} \left( \sqrt{\alpha} + \frac{\alpha}{\sqrt{\alpha}} \right) = \sqrt{\alpha}
\end{align*}


\noindent
\underline{Fall 3:} $x_1 > \sqrt{\alpha} \Rightarrow x_1 = \sqrt{\alpha} + h_1$, wobei $h_1 > 0$ und $h_1 < \sqrt\alpha$
\begin{align*}
\Rightarrow x_2 &= \frac{1}{2} \left( \sqrt{\alpha} + h_1 + \frac{\alpha}{\sqrt{\alpha} + h_1} \right) \\
& = \frac{1}{2} \left(\frac{\left(\sqrt{\alpha} + h_1\right)^2}{\sqrt{\alpha} + h_1} + \frac{\alpha}{\sqrt{\alpha} + h_1} \right) \\
&= \frac{1}{2} \left(\frac{\alpha + 2\sqrt{\alpha} h_1 + h_1^2 + \alpha}{\sqrt{\alpha} + h_1} \right) \\
&= \frac{1}{2} \left(\frac{2\alpha + 2\sqrt{\alpha} h_1 + h_1^2}{\sqrt{\alpha} + h_1} \right) \\
&= \frac{1}{2} 2\sqrt{\alpha} \frac{\sqrt{\alpha} + h_1 + \frac{h_1^2}{2\sqrt{\alpha}}}{\sqrt{\alpha} + h_1} \\
&= \sqrt{\alpha} \left(1 + \frac{h_1^2}{2\sqrt{\alpha}\left(\sqrt{\alpha} + h_1\right)}\right) \\
\Rightarrow x_2 &> \sqrt{\alpha}
\end{align*}
Betrachte nun beliebige $n \geq 2$. Per Induktion wird gezeigt, dass $x_n > \sqrt{\alpha}$ ist:
\begin{align*}
    \text{Induktionsanfang:}&\quad x_2 > \sqrt{\alpha} \quad(\text{ist erfüllt, siehe oben}) \\
    \text{Induktionsvoraussetzung:} & \quad \text{für ein festes, beliebiges $n \in \mathbb{N}$, $n \geq$ gilt:}\, x_n = \sqrt{\alpha} + h_n (h_n > 0) \\
    \text{Induktionsschritt:}&\quad x_{n+1} = \frac12\left(x_n + \frac{\alpha}{x_n}\right) \\
    (\text{nach I.V.})\Rightarrow&\quad x_{n+1}= \frac12\left(\sqrt{\alpha} + h_n + \frac{\alpha}{\sqrt{\alpha} + h_n}\right) \\
    (\text{analog Fall 3})\Rightarrow&\quad x_{n+1} = \sqrt{\alpha} \left(1 + \frac{h_n^2}{2\sqrt{\alpha}\left(\sqrt{\alpha} + h_n\right)}\right) > \sqrt{\alpha},
\end{align*}
also folgt aus der Induktionsannahme für $n$, dass die Aussage für $n+1$ wahr ist. Somit gilt nach dem Satz der vollständigen Induktion $x_n > \sqrt{\alpha}$ für $n\geq 2$.

Es bleibt zu zeigen, dass $x_n$ monoton fällt für $n\geq 2$, also $x_{n+1} - x_n < 0$:
\begin{align*}
    x_{n+1} - x_n &= \frac12\left(x_n + \frac{\alpha}{x_n}\right) -x_n\\
    &= \frac12\left(\frac{\alpha}{x_n} - x_n\right) \\
    &= \frac12\frac{\alpha - x_n^2}{x_n} \\
    &= \frac{1}{2 x_n} \underbrace{\left(\sqrt{\alpha} + x_n\right)}_{>0} \underbrace{\left(\sqrt{\alpha} - x_n\right)}_{<0} \\
    \Rightarrow x_{n+1} - x_n < 0,
\end{align*}
also ist $x_n$ monoton fallend.

Sei Der Grenzwert $\lim x_n \coloneqq x$. Dann ist auch $\lim x_{n+1} = x$
\begin{align*}
    \Rightarrow \lim x_{n+1} &= \frac12 \left(\lim x_n + \frac{\alpha}{\lim x_n}\right) \\
    \Rightarrow x &= \frac12 \left(x + \frac{\alpha}{x}\right) \\
    \Leftrightarrow 2x &= x + \frac{\alpha}{x} \Leftrightarrow x = \frac{\alpha}{x} \Leftrightarrow x^2 = \alpha \Leftrightarrow x = \sqrt{\alpha} \\
    \Rightarrow \lim x_n &= \sqrt{\alpha}
\end{align*}
Die Reihe konvergiert also gegen $\sqrt{\alpha}$.
\end{lsg}



