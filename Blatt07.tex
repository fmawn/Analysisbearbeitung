\section{Blatt}

\begin{aufg}[6 Punkte]
Beweisen Sie Satz~3.42.
\end{aufg}

\bigskip

\begin{lsg}
\end{lsg}

\bigskip


\begin{aufg}[6 Punkte]
Bestimmen Sie alle H\"aufungswerte der Folge~$(x_n)_{n\in\N}$ mit 
\[
 x_n \coloneqq (-1)^{\lfloor \frac{n}{2}\rfloor} \left( 7 + (-1)^n\left(1+\frac1n\right)^{n+1} \right)\,. 
\]
Bestimmen Sie au{\ss}erdem $\limsup x_n$ und $\liminf x_n$.
\end{aufg}

\bigskip

\begin{lsg}
\end{lsg}

\bigskip

\begin{aufg}[6 Punkte]
Es sei $(a_n)_{n\in\N}$ eine beschr\"ankte Folge in~$\R$. Wir definieren die Folge~$(b_n)_{n\in\N}$ der \emph{arithmetischen Mittel} durch
\[
 b_n \coloneqq \frac1n\sum_{k=1}^n a_k\,.
\]
Zeigen Sie:
\begin{enumerate}[label=$\mathrm{(\roman*)}$, ref=$\mathrm{\roman*}$]
\item Es gilt $\liminf a_n \leq \liminf b_n \leq \limsup b_n \leq \limsup a_n$.
\item Wenn $(a_n)_n$ konvergiert, dann konvergiert auch $(b_n)$. Was ist dann der Grenzwert von $(b_n)$?
\end{enumerate}
\end{aufg}
 
\bigskip

\begin{lsg}\mbox{ }
\begin{enumerate}[label=$\mathrm{(\roman*)}$, ref=$\mathrm{\roman*}$]
\item 
\end{enumerate}
\end{lsg}

\bigskip

\begin{aufg}[6 Punkte]
Es seien $\alpha$ und $x_1$ reelle \textbf{positive} Zahlen. Weiterhin sei die Folge~$(x_n)_{n\in\N}$ definiert durch 
\[
 x_{n+1} \coloneqq \frac12\left(x_n + \frac{\alpha}{x_n}\right) \quad\text{f\"ur $n\in\N$.}
\]
Zeigen Sie, dass die Folge $(x_n)_n$ konvergiert und bestimmen Sie ihren Grenzwert.
\end{aufg}

\bigskip

\begin{lsg}  
\end{lsg}



