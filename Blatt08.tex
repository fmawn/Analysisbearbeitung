\section{Blatt}

\begin{aufg}[6 Punkte]
\glqq Ist schon Weihnachten \ldots\grqq, denkt sich der Grinch, der nichts mehr hasst als das Fest der Harmonie. Aus Rache will er aus der harmonischen Reihe alle Summanden, die die Ziffer Null enthalten, streichen und hofft, damit die Welt in Chaos zu st\"urzen. Sein Vorhaben kann nur aufgehalten werden, wenn die so entstandene \emph{unharmonische} Reihe
\[
 \frac{1}{1} + \frac{1}{2} + \frac{1}{3} + \ldots + \frac{1}{9} + \frac{1}{11} + \ldots + \frac{1}{19} + \frac{1}{21} + \ldots + \frac{1}{29} + \frac{1}{31} + \ldots
\]
konvergiert. Konvergiert oder divergiert die Reihe?
\end{aufg}

\bigskip

\begin{lsg}
\end{lsg}


\bigskip


\begin{aufg}[6 Punkte]
\"Uberpr\"ufen Sie, ob die folgenden Reihen konvergieren oder divergieren.
\begin{enumerate}[label=$\mathrm{(\roman*)}$, ref=$\mathrm{\roman*}$]
\item $\sum\limits_{k=1}^\infty \frac{1}{k^2}$
\item $\sum\limits_{n=1}^\infty \frac{n^4}{2^n}$
\item $\sum\limits_{p=5}^\infty \binom{p+2}{p}^{-\frac1p}$
\item $\sum\limits_{n=2}^\infty \frac1n \left( \sqrt[n]{n} - \sqrt[n+1]{n+1} \right)$
\item $\sum\limits_{q=1}^\infty (-1)^{q+1} \frac{\sqrt[q]{q}}{q}$
\item $\sum\limits_{n=100}^\infty \frac{1}{\sqrt{n!}}$
\end{enumerate}
\end{aufg}

\bigskip

\begin{lsg}\mbox{ }
\begin{enumerate}[label=$\mathrm{(\roman*)}$, ref=$\mathrm{\roman*}$]
\item 
\end{enumerate}
\end{lsg}

\bigskip

\begin{aufg}[6 Punkte]
Gegeben sei die Reihe $\sum\limits_{n=0}^\infty 2^{(-1)^n-n}$. Zeigen Sie:
\begin{enumerate}[label=$\mathrm{(\roman*)}$, ref=$\mathrm{\roman*}$]
\item Die Reihe konvergiert; geben Sie eine konvergente Majorante an.
\item Die Konvergenz der Reihe ergibt sich auch mit Hilfe des Wurzelkriteriums.
\item Das Quotientenkriterium gibt keine Information \"uber das Konvergenzverhalten.
\end{enumerate}
\end{aufg}
 
\bigskip

\begin{lsg}\mbox{ }
\begin{enumerate}[label=$\mathrm{(\roman*)}$, ref=$\mathrm{\roman*}$]
\item 
\end{enumerate}
\end{lsg}


\bigskip


\begin{aufg}[6 Punkte]
Es seien $(a_n)_n$ und $(b_n)_n$ Folgen in~$\R^+$ und es existiere 
\[
 \gamma\coloneqq \lim_{n\to\infty} \frac{a_n}{b_n} > 0\,.
\]
Zeigen Sie: die Reihen $\sum a_n$ und $\sum b_n$ haben dasselbe Konvergenzverhalten.
\end{aufg}


\bigskip

\begin{lsg}  
\end{lsg}


\bigskip

\begin{aufg}[2 Punkte; Sonderaufgabe; wird fortgesetzt, Anleitung lesen; separate Abgabe]\label{aufg:sonder1}
Es sei $M\subseteq\R$, $M\not=\emptyset$ und $x>0$ f\"ur alle $x\in\M$. Wir setzen
\[
 \frac1M \coloneqq \left\{ \frac1x \in \R \colon x\in M \right\}\,.
\]
Zeigen Sie: Ist $\inf M >0$, dann gilt 
\[
 \sup \frac1M = \frac{1}{\inf M}\,.
\]
\end{aufg}

\bigskip

\begin{lsg}
\end{lsg}

\bigskip 

\begin{center}
 {\large\textbf{Untenstehende Anleitung zur Sonderaufgabe beachten!}}
\end{center}

\bigskip


\noindent
\textbf{Anleitung zur Sonderaufgabe:} Mit Aufgabe~\ref{aufg:sonder1} machen wir ein Experiment, das Sie n\"achste Woche erfahren werden. Die Aufgabe wird also fortgesetzt. Was m\"ussen Sie im Moment tun?

Jede Abgabegruppe bearbeitet diese Aufgabe bitte mit h\"ochster Priorit\"at und versucht, eine m\"oglichst gute L\"osung aufzuschreiben. Diese Aufgabe schreiben Sie bitte auf ein separates Blatt oder eine separate Datei und senden Sie an Ihre \"Ubungsleiter*innen in einer separaten Datei. Versehen Sie die Abgabe mit Ihren Namen, aber \textbf{nicht} mit Ihrem Matrikelnummern. 

Es ist wichtig, dass jede Gruppe eine Abgabe einreicht. Wenn Sie die Aufgabe nicht vollst\"andig l\"osen k\"onnen, geben Sie bitte eine Teill\"osung oder Erkl\"arung Ihrer Versuche und Ans\"atze ab.

Am 14.12.\@ erfahren Sie, wie das Experiment weitergeht.

