\section{Blatt}

\begin{aufg}[6 Punkte]
Zeigen Sie:
\begin{enumerate}[label=$\mathrm{(\roman*)}$, ref=$\mathrm{\roman*}$]
\item Die Reihe
\[
 \sum_{n=1}^\infty \ln\left( 1 + \frac{1}{n^p} \right)
\]
divergiert f\"ur $p=1$, aber konvergiert f\"ur $p=2$.
\item Die Funktion 
\[
 f\colon (0,\infty)\to\R\,,\quad x\mapsto x + e^{-x} - C
\]
besitzt f\"ur jedes $C>1$ eine Nullstelle. Was passiert f\"ur $C=1$ und was f\"ur $C<1$?
\item Jedes reelle Polynom vom Grad~$3$ besitzt mindestens eine reelle Nullstelle.
\end{enumerate}
\end{aufg}


\bigskip

\begin{lsg}\mbox{ }
\begin{enumerate}[label=$\mathrm{(\roman*)}$, ref=$\mathrm{\roman*}$]
\item 
\end{enumerate}
\end{lsg}


\bigskip


\begin{aufg}[6 Punkte]
Sei die Funktion $f\colon\R\rightarrow\R$ definiert durch 
\[
f(x)\coloneqq \left| \left\lfloor x+\frac{1}{2} \right\rfloor -x \right|\,.
\]
Zeichnen Sie den Graphen der Funktion $f$ in ein geeignetes Koordinatensystem und zeigen Sie:
\begin{enumerate}[label=$\mathrm{(\roman*)}$, ref=$\mathrm{\roman*}$]
    \item F\"ur alle $x\in\R$ gilt $0\le f(x) \le \frac{1}{2}$.
    \item F\"ur alle $x\in \R$ und $n\in\Z$ gilt $f(x+n)=f(x)$.
    \item Die Funktion $f$ ist stetig.
\end{enumerate}
\end{aufg}

\bigskip

\begin{lsg}
\end{lsg}


\bigskip

\begin{aufg}[6 Punkte]
Berechnen Sie den punktweisen Grenzwert der folgenden Funktionenfolgen~$(f_n)_{n\in\N}$ und entscheiden Sie, ob die Konvergenz gleichm\"a{\ss}ig ist:
\begin{enumerate}[label=$\mathrm{(\roman*)}$, ref=$\mathrm{\roman*}$]
\item $f_n(x) = \begin{cases} 0 & x \leq n \\ x-n & x > n \end{cases}$ auf
${}]-\infty, 196560]$ und auf $\R$.
\item $f_n(x) = \frac{x}{1+(nx)^2}$ auf $\R$.
\end{enumerate}
\end{aufg}
 
\bigskip

\begin{lsg}\mbox{ }
\begin{enumerate}[label=$\mathrm{(\roman*)}$, ref=$\mathrm{\roman*}$]
\item 
\end{enumerate}
\end{lsg}

\bigskip

\begin{aufg}[6 Punkte]\mbox{ }
\begin{enumerate}[label=$\mathrm{(\roman*)}$, ref=$\mathrm{\roman*}$]
\item Beweisen Sie folgende Aussage: Sei $(a_n)_{n\in\N}$ eine positive, monoton fallende Nullfolge. Dann konvergiert die Reihe 
\[
\sum_{n=1}^{\infty}a_n
\]
genau dann, wenn die \emph{verdichtete} Reihe 
\[
\sum_{k=0}^{\infty}2^k a_{2^k}
\]
konvergiert. 
\item Nutzen Sie diese Aussage, um das Konvergenzverhalten der Reihe 
\[
\sum_{k=1}^{\infty}\frac{1}{n^a}
\]
f\"ur $a>0$ zu untersuchen. 
\end{enumerate}
\end{aufg}

\bigskip

\begin{lsg}\mbox{ }
\begin{enumerate}[label=$\mathrm{(\roman*)}$, ref=$\mathrm{\roman*}$]
\item 
\end{enumerate}
\end{lsg}

\bigskip


\begin{aufg}[8 Punkte; Bonusaufgabe]\mbox{ }
\begin{enumerate}[label=$\mathrm{(\roman*)}$, ref=$\mathrm{\roman*}$]
\item (Fr\"uhpusher-Bonus) Laden Sie bis zum 18.01.2022, 12:00 Uhr, im git eine L\"osung zu einer Aufgabe der Bl\"atter~$1$-$9$ hoch, die bislang noch keine L\"osung hat. Damit erf\"ullen Sie auch zugleich einen Teil Ihrer Studienleistung. Wenn Sie das schon gemacht haben, erhalten Sie diese~$4$ Bonuspunkte automatisch.
%
\item F\"ur genau welche $x\in\R$ konvergiert die Reihe
\[ 
\sum_{n=1}^\infty\left(x+\frac{1}{n}\right)^n\,?
\]
\end{enumerate}
\end{aufg}


\bigskip

\begin{lsg}
\end{lsg}

