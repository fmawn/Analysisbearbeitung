\section{blatt}

\begin{aufg}[6 Punkte]
Gegeben sei die Funktion $f\colon [0,1]\to\R$ mit 
\[
 f(x) \coloneqq 
 \begin{cases}
  \frac1q & \text{f\"ur~$x>0$ mit $x=\frac{p}{q}$ mit $p,q\in\N$ 
teilerfremd},
  \\
  0 & \text{f\"ur $x$ irrational},
  \\
  1 & \text{f\"ur $x=0$}.
 \end{cases}
\]
Untersuchen Sie, in welchen Punkten $f$ stetig und in welchen Punkten unstetig 
ist.
\end{aufg}

\bigskip


\begin{lsg}
\end{lsg}


\bigskip


\begin{aufg}[6 Punkte] \mbox{ }
\begin{enumerate}[label=$\mathrm{(\roman*)}$, ref=$\mathrm{\roman*}$]
\item Die \textbf{Sinusfunktion/-reihe} ist definiert durch 
\[
 \sin(x) \coloneqq \sum_{k=0}^\infty (-1)^k \frac{x^{2k+1}}{(2k+1)!}\,.
\]
Die \textbf{Kosinusfunktion/-reihe} ist definiert durch 
\[
 \cos(x) \coloneqq \sum_{k=0}^\infty (-1)^k \frac{x^{2k}}{(2k)!}\,.
\]
Zeigen Sie, dass diese beiden Reihen auf jeder beschr\"ankten Teilmenge 
von~$\R$ gleichm\"a{\ss}ig konvergieren und dass beide Funktionen auf ganz~$\R$ 
stetig sind.
%
\item Zeigen Sie, dass die Potenzreihe
\[
 \sum_{k=0}^\infty x^k
\]
auf $(-1,1)$ nicht gleichm\"a{\ss}ig konvergiert, aber auf jeder beschr\"ankten 
Teilmenge von~$(-1,1)$.
\end{enumerate}
\end{aufg}

\bigskip


\begin{lsg}\mbox{ }
\begin{enumerate}[label=$\mathrm{(\roman*)}$, ref=$\mathrm{\roman*}$]
\item 
\end{enumerate}
\end{lsg}


\bigskip


\begin{aufg}[6 Punkte]
Berechnen Sie den Konvergenzradius der folgenden Potenzreihen
\begin{enumerate}[label=$\mathrm{(\roman*)}$, ref=$\mathrm{\roman*}$]
\item $\sum_{k=0}^\infty \frac{k+1}{3^k}x^k$
\item $\sum_{k=10}^\infty \frac{(x-12)^{2k}}{1+\frac{1}{k}}$
\item $\sum_{k=2}^\infty k\cdot (x+3)^k$
\item $\sum_{k=0}^\infty \binom{3k}{k} x^k$.
\end{enumerate}
\end{aufg}
 
\bigskip


\begin{lsg}\mbox{ }
\begin{enumerate}[label=$\mathrm{(\roman*)}$, ref=$\mathrm{\roman*}$]
\item 
\end{enumerate}
\end{lsg}


\bigskip


\begin{aufg}[6 Punkte]
Zeigen Sie, dass folgende (Anf\"ange von) Potenzreihenentwicklungen gelten (auf 
den Definitionsbereich achten!):
\begin{enumerate}[label=$\mathrm{(\roman*)}$, ref=$\mathrm{\roman*}$]
\item $\sin^3 x = x^3 -\frac12x^5 + \frac{13}{120}x^7 + \ldots$ f\"ur alle 
$x\in\R$,
\item $e^{-x}\sin x = x - x^2 + \frac13x^3 - \ldots$ f\"ur alle $x\in\R$,
\item $\frac{e^x \sin x}{\cos^2 x} = x + x^2 + \frac43 x^3 + x^4 + \ldots$ 
f\"ur hinreichend kleines~$|x|$.
\end{enumerate}
Hierbei ist $\sin^3 x \coloneqq (\sin x)^3$, analog f\"ur $\cos^2 x$.
\end{aufg}


\bigskip


\begin{lsg}\mbox{ }
\begin{enumerate}[label=$\mathrm{(\roman*)}$, ref=$\mathrm{\roman*}$]
\item
\end{enumerate}
\end{lsg}

\bigskip


\begin{aufg}[4 Punkte; Bonusaufgabe]
Untersuchen Sie die Funktionenfolge $(f_n)_n$ mit 
\[
 f_n(x) \coloneqq \sqrt[x]{(x^2+2)^2}
\]
hinsichtlich punktweiser und gleichm\"a{\ss}iger Konvergenz auf~$\R$.
\end{aufg}


\bigskip

\begin{lsg}
\end{lsg}

 


\vspace*{.5cm}


\textbf{\LARGE Wichtig:} Dieses ist das letzte \"Ubungsblatt, das in Analysis~1 
bewertet wird. Das n\"achste \"Ubungsblatt ist ein Ferienblatt, das zu Beginn 
von Analysis~2 besprochen wird (zumindest f\"ur Nicht-Lehramt). \"Uberpr\"ufen 
Sie nun folgendes f\"ur die Studienleistung:
\begin{itemize}
\item Haben Sie $50\%$ aller \"Ubungspunkte erreicht? (Die \"Ubungsleiter*innen 
k\"onnen Ihnen dabei helfen.)
\item Haben Sie zwei Aufgaben vorgerechnet?
\item Haben Sie die L\"osung einer Aufgabe ins git gepusht?
\end{itemize}
Wenn Ihnen noch etwas fehlt, versuchen Sie es noch zu schaffen. 

F\"ur die Nicht-Lehr\"amtler: Abgabe der Plenumsmappen bis 05.02.2022 (in 
Ausnahmef\"allen Verl\"angerung bis 11.02.2022). Sie k\"onnen die Plenumsmappen 
mir per Email senden oder auch ins Postfach legen (Fach~90 im MZH, Ebene~1, 
zwischen Senatssaal und Fahrst\"uhlen).

\setlength{\parindent}{0pt}

\end{document}
