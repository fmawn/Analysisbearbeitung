\documentclass[12pt,a4paper]{scrartcl}%

\usepackage{paralist} 
\usepackage{a4wide}

\usepackage[dvips]{geometry}%
\usepackage[dvips]{color}%
\usepackage{graphicx}%

\usepackage{amsfonts, amssymb, amsthm}
\usepackage[utf8x]{inputenc}
\usepackage[ngerman]{babel}
\usepackage{amssymb}
\usepackage{bbm}
\usepackage{dsfont}
\usepackage{thmtools}
\usepackage{fancyhdr}
\usepackage{mathtools}
\usepackage{wasysym}
\usepackage{microtype}

\setlength{\parindent}{0pt}

\def\rmi{\mathrm{i}}
\def\eps{\varepsilon}

\newcommand{\ii}{\,\mathrm{i}}
\newcommand{\F}{{\mathbb F}}
\newcommand{\R}{{\mathbb R}}
\newcommand{\N}{{\mathbb N}}
\newcommand{\Q}{{\mathbb Q}}
\newcommand{\Z}{{\mathbb Z}}
\newcommand{\K}{{\mathbb K}}

\newcommand{\C}{{\mathbb C}}


\geometry{left=3cm,right=3cm, top=3cm, bottom=3cm}


\begin{document}
\thispagestyle{empty}

\begin{center}
 \textbf{\LARGE Bitte lesen! Noch nicht umbl\"attern!}
\end{center}

\begin{itemize}
\item Auf der n\"achsten Seite ist ein Deckblatt f\"ur die Probeklausur. Wenn Sie das Deckblatt nicht direkt ausf\"ullen k\"onnen, machen Sie bitte eine erste Seite bei Ihrer Abgabe mit den folgenden Angaben:
\begin{center}
 Nachname, Vorname, \"Ubungsleiter*in, Studiengang 
\end{center}
\item Das Ausf\"ullen des Deckblatts z\"ahlt nicht zur Bearbeitungszeit.
\item Starten Sie bitte jede Aufgabe auf einem neuen Blatt. Wenn Sie tippen: bitte Seitenumbruch machen.
\item Ihre Bearbeitungszeit startet, sobald Sie die Aufgaben zum ersten Mal anschauen. Die Aufgaben sind zwei Seiten weiter; Sie haben 60 min zum L\"osen.
\item Erlaubte Hilfsmittel: ein DIN A4 Blatt (beidseitig), das Sie per Hand selbst beschrieben haben; keine weiteren Unterlagen
\item ACHTUNG: Der L\"osungsweg geh\"ort zur L\"osung. Ohne L\"osungsweg gibt es keine Punkte!
\item Bei Fragen: Schreiben Sie mir eine kurze Email (apohl@uni-bremen.de). 
\item Nicht schummeln! Diese Probeklausur z\"ahlt f\"ur nichts. Sie k\"onnen testen, wie gut Sie im Moment sind.
\end{itemize}

\bigskip

\textbf{Nach der Probeklausur:}
\begin{itemize}
\item Schicken Sie Ihre L\"osungen als pdf-File per Email an mich (apohl@uni-bremen.de).
\item Falls Sie Wissensl\"ucken bemerkt haben, wiederholen Sie den Stoff und l\"osen Sie die Haus\"ubungen zu dem Thema nochmals.
\item \"Uberlegen Sie, ob Sie Ihren \glqq Spickzettel\grqq{} (das DIN A4 Blatt)  noch verbessern k\"onnen.
\end{itemize}






\newpage

\thispagestyle{empty}
\textbf{Universit{\"a}t Bremen\hfill Wintersemester 2021/22}

\vspace*{12pt}
\begin{center}
\textbf{\Large Probeklausur zur Analysis~1}
\end{center}

\vspace*{12pt}
Als Hilfsmittel zugelassen ist \textbf{ein per Hand selbst (beidseitig) beschriebenes DIN A4 Blatt} und nichts weiteres. Bitte beginnen Sie f\"ur jede Aufgabe ein \textbf{neues Blatt} und beschriften Sie dieses mit Ihrem \textbf{Namen}. \textbf{Begr\"unden} Sie bei der Bearbeitung alle Ihre Antworten und L\"osungsschritte, wobei Sie die aus der Vorlesung und den \"Ubungen bekannten Resultate verwenden d\"urfen. Die Bearbeitungszeit betr\"agt \textbf{60 Minuten}.

\vspace*{24pt}
\begin{table}[h]
    \centering
    \begin{tabular}{l l}
    \textbf{Nachname (hier und auf jeder Seite):} & \line(1,0){200} \\ & \\
    \textbf{Vorname (hier und auf jeder Seite):} & \line(1,0){200} \\ & \\
    \textbf{\"Ubungsleiter*in:} & \line(1,0){200} \\ & \\
    \textbf{Studiengang:} & \line(1,0){200}
    \end{tabular}
\end{table}


\vspace*{48pt}
\begin{table}[h]
    \centering
    \begin{tabular}{c|c|c|c|c|c}
    \small{Aufgabe} & 1 & 2 & 3 & 4 & $\sum$ \\ \hline
    Punkte  & \hspace{16pt} / 10 & \hspace{16pt} / 10 & \hspace{16pt} / 10 & \hspace{16pt} / 10 & \hspace{16pt} / 40
    \end{tabular}
\end{table}


\vspace*{3cm}
\begin{center}
 Die Bearbeitungszeit startet, sobald Sie umbl\"attern. 
\end{center}



\begin{center}
 \textbf{Viel Erfolg!}
\end{center}




\newpage
\thispagestyle{empty}


\textbf{Aufgabe 1 (10 Punkte):} 
\begin{enumerate}[(i)]
    \item Finden Sie eine Darstellung der Form $z=a+\rmi b$ mit $a, b \in\R$ f\"ur \begingroup
    \large
    \begin{equation*}
    \frac{5-\rmi}{1-\rmi-2\rmi^2+2\rmi^3}\,.
    \end{equation*}
    \endgroup
    \item Zeigen Sie: Es existiert ein $N\in\N$ so, dass 
    \begingroup
    \large
    \begin{equation*}
    \forall\, n\in \N, \; n\geq N: \quad n!\geq 2^{n}+4n\,.
    \end{equation*}
    \endgroup
    Bestimmen Sie au{\ss}erdem das kleinste $N\in \N $, f\"ur das die obige Aussage wahr ist.
\end{enumerate}



\bigskip\bigskip 



\textbf{Aufgabe 2 (10 Punkte):} 
Bestimmen Sie Infimum und Supremum der Menge
\begingroup
    \large
    \begin{equation*}
    M =\left\lbrace -\frac{1}{2n}+n^3+(-1)^nn^3 \in \R\ \Bigg\vert\  n\in \N \right\rbrace\,.
\end{equation*}
\endgroup



\bigskip\bigskip 



\textbf{Aufgabe 3  (10 Punkte):} 
Untersuchen Sie folgende Reihen auf Konvergenz bzw.\@ Divergenz:
\begin{enumerate}[(i)]
    \item \begingroup
    \large
    \begin{equation*} \sum\limits_{n=1}^\infty\; (-1)^n \frac{n}{n^3+1}\,,
    \end{equation*}
    \endgroup
    \item \begingroup
    \large
    \begin{equation*} \sum\limits_{n=1}^\infty \frac{1}{6^n}\binom{2n}{n}\,.
    \end{equation*}
    \endgroup
\end{enumerate}



\bigskip\bigskip 




\textbf{Aufgabe 4 (10 Punkte):} 
Es sei $g\colon \R\to \R$ eine beschr\"ankte Funktion. Zeigen Sie: Die Funktion
\begingroup
\large
\[
 f\colon \R\to\R\,,\quad x\mapsto xg(x)
\]
\endgroup
ist stetig im Punkt~$0$.


\end{document}

