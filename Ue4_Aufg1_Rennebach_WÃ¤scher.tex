\documentclass{scrartcl}
\usepackage[utf8]{inputenc}
\usepackage[ngerman]{babel}
\usepackage{amsmath}
\usepackage{amssymb}
\usepackage{fontenc}

\title{Aufgabe 4.1}
\author{Lucas Rennebach, Sarah Wäscher}
\date{January 2022}

\begin{document}
	
 \maketitle
	
\section{}
\subsection*{i}

$ Z.z.: Berechne (-1 + i)^{10} $

\begin{align*}
	\\
	(-1 + i)^{10}
	\\
    = ((-1 + i)^2)^5 \text{Potenzregeln}\\
	\\
	=(1 -2i -1)^5 \text{quadrat rein multipliziert}\\
	\\
	=(-2i)^5 \text{ausgerechnet}\\
	\\
	= -32i^5
	\\
	= -32 * i^2 + i^2 +i \text{Potenzregeln}\\
	\\
	= -32 * (-1) * (-1) *i
	\\
	= -32i
\end{align*}


	
$	Z.z.: Berechne (-1 - i)^{10} $
	
	\begin{align*}
	\\
		(-1 - i)^{10}
		\\
		= ((-1 - i)^2)^5 \text{Potenzregeln}\\
		\\
		=(1 +2i +1)^5 \text{quadrat rein multipliziert}\\
		\\
		=(2i)^5 \text{ausgerechnet}\\
		\\
		= 32i^5
		\\
		= 32 * i^2 + i^2 +i \text{Potenzregeln}\\
		\\
		= 32 * (-1) *(-1) *i
		\\
		= 32i
	\end{align*}

	\subsection*{ii}
	Bild im Anhang
\end{document}
	Inhalt...
