\documentclass{scrartcl}

\title{Analysis 1: Übung 8}
\author{Dennis Oestmann, Tobias Rauer\\Tutor: Ellen Rudolph}
\date{\today}

\usepackage{mathtools}
\usepackage{enumerate}
\usepackage[ngerman]{babel}
\usepackage[utf8]{inputenc}
\usepackage[T1]{fontenc}
\usepackage{lipsum}
\usepackage{graphicx}
\usepackage{array}
\usepackage{multirow}
\usepackage{amsmath}
\usepackage{amssymb}
\usepackage{wasysym}

\newenvironment{proof}{\textit{Beweis:}}{\hfill$\square$ \\}
\newenvironment{Widerspruch}{\textit{Widerspruchsbeweis:}}{\hfill \lightning \\}

\begin{document}

\maketitle

\newpage

\section*{Aufgabe 8.2)}
	Es gilt zu überprüfen, ob die folgenden Reihen konvergieren oder divergieren.
	\begin{enumerate}[(i)]
	  
		\item $\sum^{\infty}_{k=1}\frac{1}{k^2}$ Wir versuchen die Konvergenz durch das finden einer konvergenten Majoranten zu beweisen. \\
			Abschätzung:
			\begin{align*}
				\sum^{\infty}_{k=1}\frac{1}{k^2} = \sum^{\infty}_{k=1}\frac{2}{k^2 + k^2} \leq \sum^{\infty}_{k=1}\frac{2}{k^2 + k} = \sum^{\infty}_{k=1}\frac{2}{k(k + 1)}
			\end{align*}
			Dies können wir jetzt noch umschreiben:
			\begin{align*}
				&\sum^{\infty}_{k=1}\frac{2}{k(k + 1)} \\
				= & \sum^{\infty}_{k=1}\frac{2(k-(k+1))}{k(k + 1)(k-(k+1))} \\
				= & \sum^{\infty}_{k=1}\left(\frac{2k}{k(k+1)(k-(k+1))}-\frac{2(k+1)}{k(k+1)(k-(k+1))}\right) \\
				= & \sum^{\infty}_{k=1}\left(\frac{2}{(k+1)(k-(k+1))}-\frac{2}{k(k-(k+1))}\right) \\
				= & \sum^{\infty}_{k=1}\left(\frac{2}{k^2-k(k+1)+k-(k+1)}-\frac{2}{k^2-k(k+1)}\right)
			\end{align*}
			\begin{align*}
				= & \sum^{\infty}_{k=1}\left(\frac{2}{k^2-k^2-(k+1)}-\frac{2}{k^2-k^2-k}\right) \\
				= & \sum^{\infty}_{k=1}\left(\frac{2}{-(k+1)}-\frac{2}{-k}\right) \\
				= & \sum^{\infty}_{k=1}\left(\frac{2}{k}-\frac{2}{k+1}\right)
			\end{align*}
			Nun haben wir eine Teleskopsumme, die wie folgt aussieht:
			\begin{align*}
				\sum^{\infty}_{k=1}\left(\frac{2}{k}-\frac{2}{k+1}\right) = \left(\frac{2}{1} - \frac{2}{2}\right) + \left(\frac{2}{2} - \frac{2}{3}\right) + \left(\frac{2}{3} - \frac{2}{4}\right) \pm \ldots
			\end{align*}
			Dies kann man jetzt so umklammern, dass nur noch der erste und der letzte Wert übrig bleiben.
			\begin{align*}
				\sum^{\infty}_{k=1}\left(\frac{2}{k}-\frac{2}{k+1}\right) = &\frac{2}{1} + \left(-\frac{2}{2} + \frac{2}{2}\right) + \left(-\frac{2}{3} + \frac{2}{3}\right) + \left(-\frac{2}{4} + \frac{2}{4}\right) \pm \ldots \\
				= & 2 - \lim_{k \to \infty}\frac{2}{k} = 2 - 0 = 2
			\end{align*}
			Somit haben wir für die Reihe $\sum^{\infty}_{k=1}\frac{1}{k^2}$, mit $\sum^{\infty}_{k=1}\frac{2}{k(k + 1)}$ eine konvergente Majorante gefunden, wodurch $\sum^{\infty}_{k=1}\frac{1}{k^2}$ absolut konvergent ist, und damit nach Satz 3.65 konvergent.
			
		\item $\sum^{\infty}_{n=1}\frac{n^4}{2^n}$ Anwenden des Wurzelkriteriums (W.K.).
			\begin{align*}
				a_n:=\frac{n^4}{2^n} \overset{\text{W.K.}}{\Rightarrow} &\sqrt[n]{\left|\frac{n^4}{2^n}\right|} = \sqrt[n]{\frac{n^4}{2^n}} \\
				= &\frac{\sqrt[n]{n^4}}{\sqrt[n]{2^n}} = \frac{(\sqrt[n]{n})^4}{2}
			\end{align*}
			Wir wissen nach Beispiel 3.5 (iii), dass der $\lim_{n \to \infty}\sqrt[n]{n} = 1$ ist. Daraus folgt nach Satz 3.16:
			\begin{align*}
				\lim_{n \to \infty} \frac{(\sqrt[n]{n})^4}{2} = \frac{\lim_{n \to \infty} (\sqrt[n]{n})^4}{\lim_{n \to \infty} 2} = \frac{1^4}{2} = \frac{1}{2} < 1 \\
				\Rightarrow \text{absolut konvergent} \overset{\text{Satz 3.65}}{\Longrightarrow} \text{konvergent}
			\end{align*}
		
		\item $\sum^{\infty}_{p=5}\binom{p+2}{p}^{-\frac{1}{p}}$
			\begin{align*}	
				\sum^{\infty}_{p=5}\binom{p+2}{p}^{-\frac{1}{p}} = &\sum^{\infty}_{p=5}\left(\frac{(p+2)!}{p!((p+2)-p)!}\right)^{-\frac{1}{p}} = 	\sum^{\infty}_{p=5}\left(\frac{(p+2)!}{p!\cdot 2!}\right)^{-\frac{1}{p}} \\
				= &\sum^{\infty}_{p=5}\left(\frac{(p+2)\cdot (p+1) \cdot p!}{p!\cdot 2!}\right)^{-\frac{1}{p}} = \sum^{\infty}_{p=5}\left(\frac{(p+2)\cdot (p+1)}{2!}\right)^{-\frac{1}{p}} \\
				= & \sum^{\infty}_{p=5}\left(\frac{2!}{(p+2)\cdot (p+1)}\right)^{\frac{1}{p}} = \sum^{\infty}_{p=5}\left(\frac{\sqrt[p]{2}}{\sqrt[p]{(p+2)\cdot (p+1)}}\right) \\
				= & \sum^{\infty}_{p=5}\left(\frac{\sqrt[p]{2}}{\sqrt[p]{p+2}\cdot \sqrt[p]{p+1}}\right)
			\end{align*}
			Wir betrachten nun den Grenzwert der Folge $a_p:=\left(\frac{\sqrt[p]{2}}{\sqrt[p]{p+2}\cdot \sqrt[p]{p+1}}\right)_p$:
			\begin{align*}
				\lim_{p \to \infty}\frac{\sqrt[p]{2}}{\sqrt[p]{p+2}\cdot \sqrt[p]{p+1}}
			\end{align*}
			Wir dürfen nach Satz 3.16 nun die einzelnen Grenzwerte betrachten.
			\begin{align*}
				&\frac{\lim\sqrt[p]{2}}{\lim\sqrt[p]{p+2}\cdot \lim\sqrt[p]{p+1}} \\
				= & \frac{1}{1\cdot 1} = 1
			\end{align*}
			Da es sich hier um keine Nullfolge handelt, kann die Reihe $\sum^{\infty}_{p=5}\binom{p+2}{p}^{-\frac{1}{p}}$ erst recht nicht konvergieren, also muss sie divergent sein.
			
		\item $\sum^{\infty}_{n=2}\frac{1}{n}(\sqrt[n]{n}-\sqrt[n+1]{n+1})$ Wir versuchen wieder die Konvergenz durchs finden einer konvergenten Majoranten zu zeigen. \\
			Abschätzen:
			\begin{align*}
				\sum^{\infty}_{n=2}\frac{1}{n}(\sqrt[n]{n}-\sqrt[n+1]{n+1}) \leq \sum^{\infty}_{n=2}(\sqrt[n]{n}-\sqrt[n+1]{n+1})
			\end{align*}
			Nun haben wir eine Teleskopreihe, die wie folgt aussieht:
			\begin{align*}
				\sum^{\infty}_{n=2}(\sqrt[n]{n}-\sqrt[n+1]{n+1}) = &(\sqrt{2} - \sqrt[3]{3}) + (\sqrt[3]{3} - \sqrt[4]{4}) + (\sqrt[4]{4} - \sqrt[5]{5}) + \ldots \\
				= &\sqrt{2} + (-\sqrt[3]{3}) + \sqrt[3]{3}) + (-\sqrt[4]{4} + \sqrt[4]{4}) + (-\sqrt[5]{5} + \sqrt[5]{5}) + \ldots + (-\lim_{n \to \infty} \sqrt[n]{n}) \\
				= &\sqrt{2} - \lim_{n \to \infty} \sqrt[n]{n} \\
				= &\sqrt{2} - 1
			\end{align*}
			Somit wissen wir, dass unsere Teleskopreihe nach $\sqrt{2} - 1$ konvergiert, wodurch wir eine konvergente Majorante für die Reihe $\sum^{\infty}_{n=2}\frac{1}{n}(\sqrt[n]{n}-\sqrt[n+1]{n+1})$ gefunden haben.
		
		\item $\sum^{\infty}_{q=1}(-1)^{q+1}\frac{\sqrt[q]{q}}{q}$. Dies ist eine alternierende Reihe: Wir zeigen nun, dass $(b_q)_{q > 0}:=\left(\frac{\sqrt[q]{q}}{q}\right)_q$ eine monoton fallende Nullfolge ist. Dann ist nämlich nach dem Satz 3.67 (Konvergenzkriterium von Leibniz) unsere alternierende Reihe konvergent. Wir zeigen also nun, dass der Grenzwert $0$ ist:
			\begin{align*}
				\lim_{q \to \infty}\frac{\sqrt[q]{q}}{q} \\
				= \lim_{q \to \infty}\left(\sqrt[q]{q}\cdot \frac{1}{q}\right)
			\end{align*}
			Wir wissen das sowohl $\sqrt[q]{q}$ und $\frac{1}{q}$ konvergiert, aus diesem Grund können wir mit Satz 3.16 es nun so schreiben.
			\begin{align*}
				= &\lim_{q \to \infty}\sqrt[q]{q}\cdot \lim_{q \to \infty}\frac{1}{q} \\
				= &1 \cdot 0 = 0
			\end{align*}
			Somit ist unsere Folge $(b_q)_q$ eine Nullfolge. \\
			Dadurch, dass $\sqrt[q]{q}$ ab einer ganzen Zahl von $3$ monoton fällt und, müssen wir nur noch die Fälle $q=1 und q=2$ betrachten.
			\begin{itemize}
				\item[$q=1$:]	\begin{align*}
									1 = \frac{\sqrt[1]{1}}{1} < \frac{\sqrt[1+1]{1+1}}{1+1} = \frac{\sqrt[2]{2}}{2}
								\end{align*}
								Dies stimmt, da die Quadratwurzel von 2 ungefähr bei $1,4$ liegt und dies geteilt durch $2$ kleiner als $1$ ist.
				\item[$q=2$:]	\begin{align*}
									&\frac{\sqrt[2]{2}}{2} < \frac{\sqrt[2+1]{2+1}}{2+1} = \frac{\sqrt[3]{3}}{3} \\
									\Leftrightarrow &\frac{\sqrt[2]{2^3}}{2^3} < \frac{1}{9} \\
									\Rightarrow &\frac{2 \cdot \sqrt[2]{2}}{8} \\
									\Rightarrow &\frac{\sqrt[2]{2}}{4} = \frac{\sqrt{2}}{\sqrt{2^4}} = \frac{\sqrt{2}}{(\sqrt{2})^4} \\
									= &\frac{1}{(\sqrt{2})^3} = \frac{1}{\sqrt{8}} \\
									\Rightarrow \frac{1}{\sqrt{8}} > \frac{1}{9}
								\end{align*}
								Dies stimmt auch, da $\frac{1}{9} < \frac{1}{8} < \frac{1}{\sqrt{8}}$
			\end{itemize}
			Somit ist $\sum^{\infty}_{q=1}(-1)^{q+1}\frac{\sqrt[q]{q}}{q}$ konvergent.	
			
		\item $\sum^{\infty}_{n=100}\frac{1}{\sqrt{n!}}$ Anwenden des Quotientenkriteriums (Q.K.).
			\begin{align*}
				a_n:=\frac{1}{\sqrt{n!}} \overset{\text{Q.K.}}{\Longrightarrow} \left|\frac{\sqrt{n!}}{\sqrt{(n+1)!}}\right| = \sqrt{\frac{n!}{(n+1)!}}= \sqrt{\frac{1}{n+1}}
			\end{align*}
			Da hier der Limes Inf = Limes Sup ist, reicht es den Limes von $\sqrt{\frac{1}{n+1}}$ zu betrachten.
			\begin{align*}
				\lim_{n \to \infty} \sqrt{\frac{1}{n+1}} = \sqrt{\lim_{n \to \infty} \frac{1}{n+1}} = \sqrt{0} = 0 < 1
			\end{align*}
			Somit ist die Reihe $\sum^{\infty}_{n=100}\frac{1}{\sqrt{n!}}$ konvergent.
	\end{enumerate}

\section*{Aufgabe 8.3)}

	\begin{enumerate}[(i)]
		\item Anzugeben ist die konvergente Majorante.
				\begin{align*}
					\sum^{\infty}_{n=0} 2^{(-1)^n - n} = \sum^{\infty}_{n=0} \left|\frac{2^{(-1)^n}}{2^n}\right| \leq \sum^{\infty}_{n=0} \frac{2}{2^n}
				\end{align*}
				Wenn wir nun zeigen, dass $\sum^{\infty}_{n=0} \frac{2}{2^n}$ konvergiert, dann haben wir unsere konvergente Majorante. Dafür nutzen wir das Wurzelkriterium.
				\begin{align*}
					a_n:=\frac{2}{2^n} \overset{\text{W.K.}}{\Longrightarrow} \lim_{n \to \infty}\sqrt[n]{\left|\frac{2}{2^n}\right|} = \lim_{n \to \infty}\frac{\sqrt[n]{2}}{\sqrt[n]{2^n}} = \lim_{n \to \infty}\frac{\sqrt[n]{2}}{2} = \frac{1}{2} < 1
				\end{align*}
				Da nach dem W.K. es kleiner als $1$ ist, ist $\sum^{\infty}_{n=0} \frac{2}{2^n}$ konvergent und somit eine konvergente Majorante zu $\sum^{\infty}_{n=0} \frac{2^{(-1)^n}}{2^n}$.
		\item	Konvergenz über das W.K. ermitteln.
				\begin{align*}
					a_n:= \frac{2^{(-1)^n}}{2^n} \overset{\text{W.K.}}{\Longrightarrow} \limsup_{n \to \infty}\sqrt[n]{\left|\frac{2^{(-1)^n}}{2^n}\right|}
				\end{align*}
				\begin{align*}
					\limsup_{n \to \infty}\sqrt[n]{\frac{2^{(-1)^n}}{2^n}} = \limsup_{n \to \infty}\frac{\sqrt[n]{2^{(-1)^n}}}{\sqrt[n]{2^n}} = \limsup_{n \to \infty}\frac{\sqrt[n]{2^{(-1)^n}}}{2}
				\end{align*}
				Wir betrachten nun die Fälle $n$ ist gerade und $n$ ist ungerade.
				\begin{itemize}
				\item $n$ ist gerade:
						\begin{align*}
							n = 2k ,k\in \mathbb{N} \\
							\Rightarrow &\limsup_{k \to \infty}\frac{\sqrt[2k]{2^{(-1)^{2k}}}}{2} = \limsup_{k \to \infty}\frac{\sqrt[2k]{2^{1^k}}}{2} \\
							= &\limsup_{k \to \infty}\frac{\sqrt[2k]{2}}{2} = \frac{1}{2} < 1 \Rightarrow \text{konvergent}
						\end{align*}
						
				\item $n$ ist ungerade:
						\begin{align*}
							n = 2k+1 ,k \in \mathbb{N} \\
							\Rightarrow &\limsup_{k \to \infty}\frac{\sqrt[2k+1]{2^{(-1)^{2k+1}}}}{2} = \limsup_{k \to \infty}\frac{\sqrt[2k+1]{2^{1^k \cdot (-1)}}}{2} \\
							= &\limsup_{k \to \infty}\frac{\sqrt[2k+1]{2^{(-1)}}}{2} = \limsup_{k \to \infty}\frac{\sqrt[2k+1]{\frac{1}{2}}}{2} \\
							= &\frac{1}{2} < 1 \Rightarrow \text{kovergent}
						\end{align*}
				\end{itemize}
				Hieraus folgt nach dem Einschnürungssatz, dass $\limsup_{n \to \infty}\frac{\sqrt[n]{2^{(-1)^n}}}{2} = \frac{1}{2} < 1$ ist, da $\frac{\sqrt[n]{\frac{1}{2}}}{2} \leq \frac{\sqrt[n]{2^{(-1)^n}}}{2} \leq \frac{\sqrt[n]{2}}{2}$, dies bedeutet, dass $\sum^{\infty}_{n=0} 2^{(-1)^n - n}$ konvergiert nach dem W.K..
		\item Z.z.: Das Quotientenkriterium gibt keine Information über das Konvergenzverhalten. Es muss also dies gelten:
				\begin{align*}
					\liminf\left|\frac{a_{n+1}}{a_n}\right|\leq 1 \leq \limsup\left|\frac{a_{n+1}}{a_n}\right|
				\end{align*}
				Ermitteln des Limes Sup von $\sum^{\infty}_{n=0} 2^{(-1)^n - n}$ mit Hilfe von Q.K., $a_n:=2^{(-1)^n - n}$:
				\begin{align*}
					\limsup\left|\frac{a_{n+1}}{a_n}\right| = &\limsup\left|\frac{2^{(-1)^{n+1}-(n+1)}}{2^{(-1)^n - n}}\right| = \limsup\left|\frac{2^{(-1)^{n+1}}}{2^{n+1}}\cdot \frac{2^n}{2^{(-1)^n}}\right| \\
					=& \limsup\left|\frac{2^{(-1)^{n+1}}\cdot 2^n}{2^{(-1)^n}\cdot 2^{n+1}}\right| = \limsup\left|\frac{2^{(-1)^{n+1}}}{2^{(-1)^n}\cdot 2}\right| = \limsup\left|\frac{2^{(-1)^n \cdot (-1)}}{2^{(-1)^n}\cdot 2}\right| \\
					=& \limsup\left|\frac{2^{(-1)^n \cdot (-1)}\cdot 2^{(-1)^n \cdot (-1)}}{2}\right| = \limsup\left|\frac{4^{(-1)^{n+1}}}{2}\right| = \limsup\left|\frac{4^{(-1)^{n+1}}}{2}\right|
				\end{align*}
				Es gibt jetzt 2 Möglichkeiten, entweder ist $n$ ungerade oder gerade.
				\begin{itemize}
					\item 	\begin{align*}
								&n = 2k ,k \in \mathbb{N} \\
								&\Rightarrow \lim_{k \to \infty}\left|\frac{4^{(-1)^{2k+1}}}{2}\right| = \lim_{k \to \infty}\frac{4^{(-1)}}{2}= \frac{1}{8}
							\end{align*}
					\item	\begin{align*}
								&n = 2k + 1, k \in \mathbb{N} \\
								&\Rightarrow \lim_{k \to \infty}\left|\frac{4^{(-1)^{2k+1+1}}}{2}\right| = \lim_{k \to \infty}\left|\frac{4^{(-1)^{2k+2}}}{2}\right| = \lim_{k \to \infty}\left|\frac{4^{1^k\cdot 1}}{2}\right| = \lim_{k \to \infty}\frac{4}{2} = 2
							\end{align*}
				\end{itemize}
				Da der größere Grenzwert die $2$ ist, ist auch unser $\limsup\left|\frac{4^{(-1)^{n+1}}}{2}\right| = 2 > 1$, der kleinere Grenzwert wäre dann das Limes Inf von $\left|\frac{4^{(-1)^{n+1}}}{2}\right|$. Somit ist unser Limes Inf $= \frac{1}{8} < 1$ und wir können somit keine Aussage aus dem Quotientenkriterium ziehen.
	\end{enumerate}

\end{document}